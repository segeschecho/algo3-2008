%%	SECCION documentclass																									 %%	
%%---------------------------------------------------------------------------%%
\documentclass{report}

%%---------------------------------------------------------------------------%%
%%	SECCION usepackage																											 %%	
%%---------------------------------------------------------------------------%%
\usepackage{amsmath, amsthm}
\usepackage[spanish,activeacute]{babel}
\usepackage{caratula}
\usepackage{a4wide}
\usepackage{hyperref}
\usepackage{fancyhdr}
% \usepackage{moreverb}
\usepackage{graphicx} % Para el logo magico!
\usepackage{capt-of}
\usepackage{afterpage}
\usepackage{float}
\usepackage{amssymb}
\usepackage{amsmath}
\usepackage[latin1]{inputenc}
\usepackage{subfigure}
\usepackage{algorithm}
\usepackage{algorithmic}
\usepackage[dvipsnames,usenames]{color}
\usepackage{amsfonts}
%%---------------------------------------------------------------------------%%
%%	SECCION opciones																												 %%	
%%---------------------------------------------------------------------------%%
\parskip    = 11 pt
\headheight	= 13.1pt
\pagestyle	{fancy}
\definecolor{orange}{rgb}{1,0.5,0}

\addtolength{\headwidth}{1.0in}

\addtolength{\oddsidemargin}{-0.5in}
\addtolength{\textwidth}{1.0in}
\addtolength{\topmargin}{-0.8in}
\addtolength{\textheight}{0.7in}

%%---------------------------------------------------------------------------%%
%%	SECCION document	 %%	
%%---------------------------------------------------------------------------%%
\begin{document}
\renewcommand{\chaptername}{Parte }
\renewcommand{\algorithmicrequire}{\textcolor{blue}{\textbf{Requiere:}}}
\renewcommand{\algorithmicensure}{\textbf{Asegura:}}
\renewcommand{\algorithmicend}{\textbf{Fin}}
\renewcommand{\algorithmicif}{\textcolor{blue}{\textbf{Si}}}
\renewcommand{\algorithmicthen}{\textcolor{blue}{\textbf{entonces}}}
\renewcommand{\algorithmicelse}{\textcolor{red}{\textbf{Si no}}}
\renewcommand{\algorithmicelsif }{\textcolor{blue}{\textbf{Si no y}}}
\renewcommand{\algorithmicendif}{\textcolor{blue}{\textbf{Fin si}}}
\renewcommand{\algorithmicfor}{\textcolor{Blue}{\textbf{Para}}}
\renewcommand{\algorithmicendfor}{\textcolor{blue}{\textbf{Fin para}}}
\renewcommand{\algorithmicwhile}{\textcolor{green}{\textbf{Mientras}}}
\renewcommand{\algorithmicendwhile}{\textcolor{green}{\textbf{Fin mientras}}}
\renewcommand{\algorithmicdo}{\textcolor{green}{\textbf{hacer}}}
\floatname{algorithm}{Algoritmo}

%%---- Caratula -------------------------------------------------------------%%
\materia{Algoritmos y Estructuras de Datos III (2008)}
\titulo{Trabajo Pr'actico n� 3}
\subtitulo{Dibujo incremental de grafos bipartitos}

\integrante{Gonz'alez, Emiliano}{426/06}{xjesse\_jamesx@hotmail.com}
\integrante{Mart'inez, Federico}{17/06}{federicoemartinez@gmail.com}
\integrante{Sainz-Tr'apaga, Gonzalo}{454/06}{gonzalo@sainztrapaga.com.ar}
\grupo{Grupo 3}
\resumen{
En el siguiente trabajo se explora el problema NP-completo de dibujo de grafos
bipartitos (en su variante tradicional e incremental). Se proponen algoritmos
eficientes para las problem�ticas adyacentes del conteo de cruces, as� como
un algoritmo exacto basado en la t�cnica de \textit{backtracking} y uno
heur�stico basado en GRASP. Se estudian sus tiempos de ejecuci�n y 
la calidad de los resultados obtenidos comparado con otras heur�sticas 
y con los resultados exactos.
}

% TOC, usa estilos locos
\maketitle
\pagestyle{empty}
{
\fancypagestyle{plain}
    {
    \fancyhead{}
    \fancyfoot{}
    \renewcommand{\headrulewidth}{0.0pt}
    } % clear header and footer of plain page because of ToC
\tableofcontents
}


\newpage
% arreglos los estilos para el resto del documento, y
% reseteo los numeros de pagina para que queden bien
\pagenumbering{arabic}
\fancypagestyle{plain} {
    \fancyhead[LO]{Gonz�lez, Mart�nez, Sainz Tr�paga}
    \fancyhead[C]{}
    \fancyhead[RO]{P\'agina \thepage\ de \pageref{LastPage}}
    \fancyfoot{}
    \renewcommand{\headrulewidth}{0.4pt}
}
\pagestyle{plain}
\chapter{Conteo de Cruces}
\section{Distintos algoritmos}
Dentro del problema que tenemos que resolver, un tema a tener en cuenta es el del conteo de cruces, ya que tanto la soluci�n exacta,
como las heuristicas necesitan saber muchas veces (por ejemplo para decidir donde se inserta un nodo, o si una permutaci�n ya no puede ser soluci�n) cuantos cruces hay en el dibujo. Por esta raz�n consideramos que un aspecto importante a optimizar para lograr mejorar el desempe�o de nuestros algoritmos es precisamente el conteo de cruces.

Lo primero que se puede observar es que los cruces en el dibujo depende solamente de la posici�n relativa de los nodos en las particiones. Se produce un cruce si hay dos nodos en una partici�n $v_i$, $v_j$ que estan relacionados con un $w_k$ y $w_p$ respectivamente tal que $v_i$ esta $``arriba''$ de $v_j$ y $w_k$ esta $``abajo''$ de $w_p$.

%TODO: dibujo pelotudo de esto

Podemos entonces caracterizar en que casos se producen cruces:\\
\textit{Dado un orden de los nodos $p_1=<v_1,v_2,...,v_n>$ en una partici�n y un orden $p_2=<w_1,w_2,...,w_q>$ en la otra partici�n , se produce un cruce si existen ejes $(v_i,w_j)$, $(v_k,w_p)$ tal que $i<k \wedge p<j$.}

De esta manera un primer algoritmo para contar los cruces, consiste en tomar cada par de ejes y comparar las posiciones relativas de sus nodos.
Hacer esto tiene un costo $\theta(m^2)$, siendo $m$ la cantidad de ejes, ya que la cantidad de posibles pares de ejes es $\binom{m}{2} = \frac{m*(m-1)}{2}$. Este costo puede ser excesivo, principalmente en grafos densos, donde m puede ser del orden de $n_1*n_2$, siendo $n_i$ la cantidad de nodos en las particion i.

Buscamos entonces alg�n algoritmo que nos de un orden mejor que $O(m^2)$\footnote{cuando hablamos de complejidad en el �mbito del conteo de cruces, nos referimos a complejidad en el modelo uniforme}. 

Si se ordenan los ejes del grafo primero por su componente en la particion $p_1$ y luego por su componente $p_2$, es decir si $e_i=(i_1,i_2)$ y $e_j=(j_1,j_2)$ son ejes del grafo, consideramos que $e_i < e_j \Leftrightarrow i_1 < j_1 \vee (i_1 == j_1 \wedge i_2 < j_2)$ y consideramos luego el orden en el que quedan las segundas componentes, podemos ver que cada inversi'on es un cruce. Una inversi�n en el orden $\pi=<a_1,a_2,....a_n>$ es un par $(a_i,a_j)$ tal que $a_i > a_j$ $\wedge$ $i<j$. 

Decimos que una inversi�n representa a un cruce porque como primero se ordena por la primer componente (en verdad es la componente de la primer partici�n, pero para simplificar la notaci�n nos referiremos a dicha componente como la primer componente) y luego por la segunda, si hay una inversion es porque para los ejes $(v_k,a_i)$ y $(v_p,a_j)$ val�a que $v_k < v_p \wedge a_i > a_j$, y esa es la condici�n para que se produzca un cruce.

%TODO: ver si no hay q explicar mejor como se puede calcular eso
Entonces si ordenamos los ejes de esa manera y despu�s contamos las inversiones, podemos obtener el numero de cruces. Para ordenar a los ejes se puede utilizar radix sort, suponiendo que los ejes tienen como componentes las posiciones relativas de los nodos (o que estas se pueden obtener), lo cual se puede calcular en caso de que esa informaci�n no este disponible con un costo de $n_1 + n_2 + m $, con $n_i$ la cantidad de nodos de la partici'on i, obteniendo los indices de los nodos en cada particion y luego usando esta informaci�n ``traducir'' los ejes; y luego consideramos el arreglo que tiene a las segundas componentes de los ejes como elementos, aplicar insertion sort y por cada cambio de posici�n que se hace para un elemento, contar una inversi�n y por lo tanto un cruce. Esto �ltimo tendr�a un costo $O(m+c)$, con c la cantidad de cruces. En el peor caso, todos los ejes se cruzan por lo que tendr�amos un costo $O(m^2)$ nuevamente.

Veamos un ejemplo de como funciona este algoritmo:
Consideremos el siguiente grafo bipartito:
\begin{figure}[H]
    \centering
     \includegraphics[scale=0.8]{./figuras/cruces/ejemplo1.png}
     \caption{Grafo de ejemplo para la aplicacion de los algoritmos de conteo de cruces}
     \label{ejemplo1}
\end{figure} 

Ahora ordenamos los ejes de acuerdo a su primer componente y luego a su segunda, obtenemos lo siguiente:

$$\left\langle (0,0),(1,1),(1,2),(2,0),(2,3),(2,4),(3,0),(3,2),(4,3),(5,2),(5,4)\right\rangle$$

Entonces el arreglo formado por las segundas componentes es el siguiente:

$$\left\langle0,1,2,\textcolor{Red}{0},3,4,0,2,3,2,4\right\rangle$$

Entonces, aplicamos selection sort. El 0 (de color rojo),recorre dos posiciones antes de insertarse en su posici�n correcta.

$$\left\langle0,0,1,2,3,4,\textcolor{Red}{0},2,3,2,4\right\rangle$$

Luego el tercer 0 se swapea cuatro veces:
 
$$\left\langle0,0,0,1,2,3,4,\textcolor{Red}{2},3,2,4\right\rangle$$

El segundo 2 cambia dos veces de posici�n:

$$\left\langle0,0,0,1,2,2,3,4,\textcolor{Red}{3},2,4\right\rangle$$

Ahora el segundo 3 cambia una vez de posici�n:

$$\left\langle0,0,0,1,2,2,3,3,4,\textcolor{Red}{2},4\right\rangle$$

Finalmente el �ltimo 2 es swapeado tres posiciones.

Entonces la cantidad de cruces del grafo es: $2+4+2+1+3 = 12$

Si bien en el caso general este algoritmo tiene un mejor orden, en peor caso sigue siendo $O(m^2)$.

Veamos un tercer acercamiento al problema: Sea $\sharp v_2$ la cantidad de nodos de la particion 2. Consideremos un arbol binario con $2^k$ hojas donde k es tal que $2^{k-1} < \sharp v_2 <= 2^{k}$ de modo de que cada nodo este en una hoja (podria haber hojas que no tengan a ningun nodo), dispuestos de izquierda a derecha seg�n su orden en la partici�n. Este �rbol tiene $2^{k+1}-1$ nodos. Ademas $k=\left\lceil log_2(\sharp v2)\right\rceil$

Lo que vamos a hacer es ordenar a los ejes como lo hicimos para el algoritmo anterior. Luego lo que haremos es agregar a los ejes segun dicho orden. Agregar el eje consiste en incrementar en uno el contador, en principio inicializado en 0, de la hoja correspondiente al nodo de la segunda componente del eje. Ademas se incrementa tambi�n en uno el contador del padre de dicha hoja y este incremento se va propagando hacia arriba hasta la ra�z.

Cada vez que insertamos un eje, en cada nivel, si estamos parados en un nodo izquierdo aumentamos el n�mero de cruces seg'un el valor del hermano del nodo.

El procedimiento puede resultar confuso en un principio por lo cual mostraremos un ejemplo de su aplicaci�n.
Apliquemos el algoritmo para el grafo de \ref{ejemplo1}:

Como tenemos $\sharp v_2 = 5$, el arbol va a tener 8 hojas y un total de 15 nodos.
 
\begin{figure}[H]
    \centering
     \includegraphics[scale=0.4]{./figuras/cruces/arbol.png}
     \caption{�rbol para contar cruces en el grafo de ejemplo}
     \label{arbol}
\end{figure} 

Los nodos cuadrados (del 'arbol) son las hojas que representan al nodo (del grafo bipartito) cuyo numero es el de la derecha de la hoja. Cada nodo tiene un contador , en principio inicializado en 0. El contador de cruces tambi�n esta inicializado en 0.

Recordemos que luego de ordenar los ejes el vector era:
$$\left\langle(0,0),(1,1),(1,2),(2,0),(2,3),(2,4),(3,0),(3,2),(4,3),(5,2),(5,4)\right\rangle$$

Lo primero que hacemos entonces es insertar el eje (0,0), de modo que se incrementa el contador de la hoja 0 y este incremento se propaga hacia arriba. Como 0 esta en un nodo izquierdo, lo que se hace es sumar a la cantidad de cruces el valor del contador del hermano de 0 (la hoja 1). Esto porque, porque el valor de dicho contador indica la cantidad de ejes insertados que terminaban en 1, pero ademas dado el orden que se usa para agregar a los nodos, sabemos que la primer coordenada de estos ejes que terminaban en 1 era menor que la del ejes que estoy insertando ahora, por lo tanto hay un cruce. En otras palabras, agregue un eje (a,1) antes que uno (b,0) y por como estban ordenados los nodos, sabemos que $a<b$ y como $1>0$, hay una inversi�n.

Al subir de nivel, como tambi�n estamos en un nodo izquierdo, agregamos a la cantidad de cruces el valor del contador del hermano correspondiente. En este caso, dicho contador guarda la cantidad de ejes agregados que terminan en 2 0 3. Y asi sucesivamente hasta la ra�z. 

Luego de esta inserci�n obtenemos el siguiente �rbol:
\begin{figure}[H]
    \centering
     \includegraphics[scale=0.4]{./figuras/cruces/arbol2.png}
     \caption{�rbol con el eje (0,0) insertado}
     \label{arbol2}
\end{figure} 

De manera analoga, se insertan los ejes (1,1) y (1,2), sin que se generen cruces, en este caso el arbol queda de la siguiente manera:

\begin{figure}[H]
    \centering
     \includegraphics[scale=0.4]{./figuras/cruces/arbol3.png}
     \caption{�rbol con los ejes (0,0),(1,1) y (1,2) insertados}
     \label{arbol3}
\end{figure} 

Luego corresponde insertar el eje (2,0). Incrementamos en uno el contador de la hoja 0. Como es un nodo izquierdo, sumamos a cantidad de cruces el valor de la hoja 1, que es 1. Este incremento corresponde al cruce del eje (2,0) con el eje (1,1). Subimos un nivel e incrementamos en 1 el contador del padre de la hoja 0. Nuevamente sumamos al contador de cruces, el valor del contador del hermano del nodo donde estamos parados, ya que otra vez estamos en un nodo izquierdo. Este nuevo incremento corresponde al cruce entre el eje (2,0) y el eje (1,2). Seguimos subiendo hasta llegar a la ra�z, incrementando el valor de los contadores y en el caso de pasar por un nodo izquierdo, sumando el valor de los contadores de los nodos derechos. En particular en este caso, volvemos a caer en un nodo izquierdo, pero el contador de su hermano es 0. Esto se debe a que todavia no insertamos ning�n eje que terminara en 4.

\begin{figure}[H]
    \centering
     \includegraphics[scale=0.4]{./figuras/cruces/arbol4.png}
     \caption{�rbol con los ejes (0,0),(1,1),(1,2) y (2,0) insertados}
     \label{arbol3}
\end{figure} 


El procedimiento se repite hasta colocar todos los ejes.

Veamos que costo tiene este algoritmo:\\ 
Primero ordenamos los ejes, como lo hacemos con radix sort, el costo es $max(m,\sharp v_1, \sharp v_2)$, puesto que tengo que ordenar m ejes, y para hacerlo voy a necesitar de $\sharp v_1$ $buckets$ primero y $\sharp v_2$ $buckets$ luego.
Una vez ordenado y armado el 'arbol, que se puede hacer en O($v_2$) como se vera mas adelante, vamos a mirar todos los ejes. Para cada eje recorremos el arbol desde una hoja hasta la raiz. Como el arbol tiene $2^k$ hojas, tiene $log_2(2^k)$ de altura, pero $log_2(2^k)=k=\left\lceil log_2(\sharp v2)\right\rceil$.

El arbol se puede implementar sobre un arreglo, de modo que la posici�n 0 sea la ra�z, la 1 y 2 sus hijos izquierdo y derecho respectivamente, y asi sucesivamente. De esta manera, los nodos izquierdos del �rbol quedan en las posiciones impares, y aumentar cada contador, asi como tambi�n moverse dentro del �rbol de un nodo a su padre se puede hacer en O(1). Por lo tanto, el costo de insertar todos los ejes es $O(max(m,\sharp v_1, \sharp v_2)+m*log(\sharp v_2) )$

Ahora bien, en vez de ordenar primero por la primer componente y luego con respecto a la segunda, podr'iamos hacer el mismo procedimiento pero ordenando primero por la segunda, luego por la primera y armando el arbol para $v_1$. Entonces, en particular el procedimiento se podria realizar utilizando el  $v_i$ de menor cardinal. Con lo cual el costo de las inserciones es de $O(max(m,\sharp v_1, \sharp v_2)+m*log(min_{i=1,2}(\sharp v_i) ))$

Repasando tenemos 3 algoritmos para calcular los cruces:
\begin{itemize}
\item El primero, consiste en revisar todo par de ejes. Tiene un costo $O(m^2)$
\item El segundo ordena los ejes y luego realiza insertion sort para contar inversiones. Tiene un costo de $O(max(m,\sharp v_1, \sharp v_2)+c)$
\item El tercero utiliza el arbol binario para contar inversiones. Tiene un orden $O(max(m,\sharp v_1, \sharp v_2)+m*log(min_{i=1,2}(\sharp(v_i))))$
\end{itemize}

Todas estos algoritmos requieren conocer el ``orden'' de los nodos en cada particion, lo cual podria hacerse, en $O(\sharp(v_1) + \sharp(v_2))$ costo que se suma a los algoritmos en caso de que no se tenga dicha informaci�n.

Si $m > log(min_{i=1,2}(\sharp(v_i)))$ combiene utilizar el tercer algoritmo, pues tiene una complejidad menor que la de los otros dos.

Si en cambio $m < log(min_{i=1,2}(\sharp(v_i)))$ (un grafo con muy pocos ejes) resulta conveniente utilizar el segundo algoritmo ya que provee un mejor orden.

Sin embargo, podemos evitar los casos donde se tengan pocos ejes. Para hacerlo, preprocesamos el grafo, de modo de sacar los nodos aislados, ya que estos se pueden insertar en cualquier lado sin que agreguen cruces.En el caso de un nodo del grafo original, si bien no puede insertarse en cualquier lado en el sentido estricto, vale que no suma cruces igualmente, por lo cual se lo puede sacar para aplicar cualquier algoritmo y luego insertarlo en la soluci�n en una posici�n valida.

De esta manera, el algoritmo 3 se muestra como la mejor opci�n.

\section{Reutilizaci�n de los c�lculos}
\label{reUso}
Si bien hay situaciones donde el orden relativo de los nodos en las particiones se modifica sustancialmente, por lo cual es necesario recalcular los cruces nuevamente, lo cual, como vimos en la secci�n anterior, tiene un costo bastante alto; hay otros casos donde se hace una modificaci�n mas peque�a al orden de los mismos y es posible no recalcular todos los cruces.

En particular si tenemos un orden de los nodos $\pi=\left\langle v_1,v_2,...v_i,v_{i+1},...,v_n \right\rangle$ y realizamos un ``swap'' entre dos posiciones consecutivas $i$, $i + 1$, podemos observar que si $\pi_1=C(\left\langle v_1,v_2,...,v_{i+1},v_{i},...v_n \right\rangle$ y definimos C($\pi$,$\rho$) como la cantidad de cruces entre los ejes del grafo, dado que los nodos de la primer partici�n estan ordenados seg'un $\pi$ y los de la segunda segunda seg�n $\rho$, vale que:

 $$C(\pi_1,\rho) = C(\pi,\rho) - CrucesEntre(v_i,v_{i+1},\rho) + CrucesEntre(v_{i+1},v_i,\rho)$$

Donde CrucesEntre(a,b,$\rho$) es la cantidad de cruces entre ejes de $a$ y ejes de $b$ si $a$ esta en una posicion relativa menor que la de $b$ y dado que los nodos de la otra partici�n estan en el orden $\rho$. Esto se debe a que como dijimos anteriormente, los cruces dependen solo del orden relativo. Entonces si intercambiamos dos posiciones consecutivas, el orden relativo de los demas nodos se mantiene, es decir, los que estaban ``abajo'' de ellos siguen estando all�, y los que estaban ``arriba'' tambi�n, de modo que solo cambian los cruces que hay entre los dos nodos movidos.

\begin{figure}[H]
    \centering
    \subfigure[]{
     \includegraphics[scale=0.2]{./figuras/cruces/crucesPreSwap.png}}
     \subfigure[]{
\includegraphics[scale=0.2]{./figuras/cruces/crucesPostSwap.png} }
     \label{fig:swap}
     \caption{Al intercambiar los nodos verde y azul, los cruces que involucran al resto de los nodos se mantienen}
\end{figure} 


Para calcular crucesEntre se puede utilizar el algoritmo antes descripto para el conteo de cruces en general, simulando una partici�n que solo contenga a los nodos $a$ y $b$. En este caso, como la partici�n mas chica tiene 2 nodos (o menos si una partici�n era solo un nodo), resulta que el algoritmo puede calcular los cruces en $O(\sharp v_2 + m_{a} + m_{b})$ con $m_{a,b}$ la cantidad de ejes incidentes a $a$ y a $b$ y $v_2$ es la otra partici�n. 

En el caso en que a y b tengan pocos ejes, de modo tal que $m_{a}*m_{b} < \sharp v_2$ , utilizamos la versi�n simple del algoritmo toma todos los pares de ejes de a y b.
 
En general si se intercambian dos nodos de posiciones consecutivas, $i$, $j$, la nueva cantidad de cruces se puede calcular solo teniendo en cuenta el fragmento de la particion entre las posiciones $i$ y $j$ (o $j$ e $i$ si $j<i$). Es decir si  $\pi=\left\langle v_1,v_2,...v_i,...,v_{j},...,v_n \right\rangle$ y $\pi_1=\left\langle  v_1,v_2,...v_j,...,v_i,...,v_n \right\rangle$, vale que

$$C(\pi_1,\rho) = C(\pi,\rho) - \sum_{k=i+1}^{j-1}{(CrucesEntre(v_i,v_k,\rho) + CrucesEntre(v_k,v_j,\rho))}$$
$$-CrucesEntre(v_i,v_j,\rho) +  \sum_{k=i+1}^{j-1}{(CrucesEntre(v_k,v_i,\rho) + CrucesEntre(v_j,v_k,\rho))}$$
$$+CrucesEntre(v_j,v_i,\rho)$$


Por lo tanto en estos casos se puede hacer algo un poco mejor que recalcular todo de nuevo. 

Para averiguar $\sum_{k=i+1}^{j-1}{(CrucesEntre(v_i,v_k,\rho))}$, se puede considerar un $\pi_3 = <v_i,w>$ donde w consiste en ``juntar'' todos los nodos $v_k$ con $k = i +1 ... j-1$, dandole todos sus ejes (podr�amos obtener un multigrafo bipartito). 
%FIXME: calcular orden
Hecho esto, aplicar el algoritmo de conteo de cruces para $\pi_3$ y como $\pi_3$ tiene 2 nodos, el algoritmo es $O(m + \sharp v1 + \sharp v2 + m)$.  
%TODO: completar explicando como. 
%TODO: agregar dibujos
\begin{figure}[H]
    \centering
    \subfigure[]{
     \includegraphics[scale=0.2]{./figuras/cruces/nuevoNodo.png}}
     \subfigure[]{
\includegraphics[scale=0.2]{./figuras/cruces/compactado.png} }
     \label{fig:swap}
     \caption{Compactaci�n del grafo para averiguar cuantos cruces agrega un nuevo nodo}
\end{figure} 

Otro caso donde se puede evitar re calcular todos los cruces es cuando se agrega un nodo al grafo, particularmente si se agrega al principio de la partici�n. En este caso, los cruces existentes entre otros nodos se mantienen, solo se podr�an agregar nuevos cruces con los ejes del nodo recien agregado. Entonces podemos aplicar una estrategia similar al caso anterior, y colapsar los nodos de la partici�n donde se esta agregando el nuevo nodo, dejando solo al nuevo nodo y a un nodo $w$ con todos los ejes del resto, de modo de que el algoritmo sea O(m).






   

\chapter{Algoritmo exacto}
\section{Desarrollo}
Un dibujo incremental valido para el problema, consiste en una permutaci�n de los nodos de cada partci�n que mantenga el orden relativo de los nodos antiguos. Si tenemos $v_i$ nodos inicialmente en la particion i, y se agregan $IV_i$ nodos en cada partici�n, la cantidad posible de soluciones es:
$$IV_1!*\dbinom{IV_1 + v_1}{v_1}*IV_2\dbinom{IV_2 + v_2}{v_2}$$
Esto se debe a que la partici�n 1 tiene $IV_1 + v_1$ nodos, por lo tanto hay esa cantidad de posiciones. De esas tengo que usar $v_1$ para los nodos viejos. Una vez que elegimos las posiciones, el orden entre ellos es fijo, por lo que si elegimos $v_1$ posiciones, solo hay 1 forma de ubicarlos. Pero en las $IV_1$ posiciones que quedan libres podemos poner cualquier permutaci�n de los nodos nuevos. De ahi viene que la cantidad de ordenes v�lidos para la particion 1 sea: 
$$ IV_1!*\dbinom{IV_1 + v_1}{v_1} $$
Para cada una de estas permutaciones, tenemos, haciendo un razonamiento similar a lo visto antes, una cantidad de permutaciones igual a:
$$IV_2!*\dbinom{IV_2 + v_2}{v_2}$$ permutaciones en la segunda partici�n.  

Dada la naturaleza del problema a resolver, lo que planteamos como acercamiento para un algoritmo exacto es utilizar la tecnica de backtracking.

La idea es obtener una soluci�n base de alguna manera, por ejemplo a partir de la heuristica constructiva que desarrollamos mas adelante en la parte \ref{constructivas}, y usar esta soluci�n inicial para poder descartar de las posibles permutaciones de los nodos a las que generan mas cruces, y cada vez que encontramos una permutaci�n que genera menos cruces, utilizamos esa para filtrar al resto. Para armar las soluciones, partimos de los nodos antiguos ya puestos, y vamos llenando la partici�n que tenga menos permutaciones validas (en adelante, le diremos partici�n 1), cuando que llenamos esta, probamos todas las permutaciones de la otra.

De esta manera, el �rbol de backtracking que tenemos es:
\begin{figure}[H]
\centering
\setcounter{subfigure}{0}
\includegraphics[scale=0.25]{./figuras/exacto/arbolbt.png}
\caption{Arbol de backtracking}
\end{figure}

Nuestro algoritmo lo va recorriendo a la manera de DFS, cortando aquellas ramas que se pueden descartar por tener un n�mero de cruces mas alto que la mejor soluci�n encontrada hasta el momento.

\subsection{Podas}
La poda basica que efectuamos consiste en eliminar aquellas permutaciones que sabemos que no pueden ser soluci�n porque ya tienen mas cruces que otra permutaci�n que armamos. 

Sin embargo, nos gustar�a poder desarrollar alg�n otro tipo de poda, que nos permita mejorar el rendimiento del algoritmo.

Consideremos un dibujo con dos permutaciones $V = <v_1,v_2,...,v_n>$, $W = <w_1,w_2,...,w_k>$, la cantidad de cruces se puede obtener como:

$$Cruces(V,W) = \sum_{i=1}^{k-1}{\sum_{j=i+1}^{k}{crucesEntre( w_i,w_j,V)}}$$

Entonces, una vez que tenemos llena la primer partici�n, podemos obtener una cota inferior para la cantidad de cruces que va a haber en la partici�n 2:

$$Cruces(V,W) \geq \sum_{i=1}^{k-1}{\sum_{j=i+1}^{k}{min(crucesEntre(w_i,w_j,V),crucesEntre(w_j,w_i,V))}}$$

Esto nos dice que dados dos nodos, $w_i$ y $w_j$, sabiendo en que orden estan los nodos de la otra partici�n, hay dos formas de coloarlos y una forma que genera menos cruces (podrian ser las dos formas si son iguales). Entonces en el mejor de los casos, todos los nodos estan ordenados de modo que dado un par cualquiera cumplen que estan en el mejor orden. 

De esta manera, obtenemos una cota que nos permite desarrollar otra poda para aplicar a nuestro algoritmo: Una vez que completamos la primer partici�n, podemos ver antes de agregar un nodo a la segunda partici�n, si los cruces que ya hay entre los nodos puestos, mas este limite inferior para el n�mero de cruces que involucran a los nodos que todavia no pusimos, es mayor que la cantidad de la mejor soluci�n hasta el momento, no vale la pena seguir construyendo esta permutaci�n.

Es de notar que el calculo de esta cota implica un overhead, por eso un factor que deberemos estudiar es si es conveniente aplicarla, ya que en el peor caso, no se poda nunca y el rendimiento es peor que si no usaramos la poda. Entonces algo que debemos observar es si la poda aporta un beneficio o es mas bien un overhead inutil.

\subsection{Tabulado de resultados}
Para ir construyendo las permutaciones, vamos agregando nodos al final, hacemos una llamada recursiva donde agregamos a los otros nodos. Cuando termina una llamada, lo que hacemos es un swap entre el nodo que agregamos y su anterior, y se repite la llamada, asi hasta que el nodo llega al final de la partici�n y hay que sacarlo.

Cada vez que hacemos un swap, calculamos la cantidad de cruces que hay entre dos nodos antes de hacer el swap y despues de hacerlo, para saber cuantos cruces hay en el dibujo que estamos armando.

Si consideramos la primer partici�n, como la armamos primero, resulta que en la partici�n opuesta solo estan los nodos antiguos, cuyo orden es fijo. Por lo cual los cruces entre dos nodos de la primer partici�n, cuando esta todavia no se lleno es siempre la misma. 

Esto quiere decir, que estamos repitiendo calculos, ya que al sacar un nodo y volverlo a poner, calculamos los cruces que genera con los demas, lo cual podriamos haber evitado. Por esta razon, al comenzar el backtracking, calculamos la cantidad de cruces entre dos nodos cualesquiera de la primer partici�n, de modo tal que cuando lo estemos moviendo mediante swaps, no necesitemos calcular nuevamente los cruces que genera con sus compa�eros de partici�n.

Si bien hacer esto, tiene un costo espacial y temporal $O(V_1^2)$, dado el costo que tiene contar los cruces, y la cantidad de veces que es necesario hacerlo, creemos que es una optimizaci�n valida.

De manera similar, cuando se llena la partici�n 1, los cruces entre nodos de otra partici�n son siempre iguales, por esta raz�n cada vez que llenamos a la primer partici�n, calculamos una tabla similar para la segunda. Esta tabla ademas nos es de mucha utilidad para poder calcular la poda antes descripta en un orden mucho menor, ya que todos los valores que necesitamos sumar, ya estan calculados.
 
\section{Pseudocodigo}
\begin{algorithm}[H]
\caption{Resuelve de forma exacta el problema de dibujar grafos incrementales bipartitos}
\begin{algorithmic}[1]
\STATE generar excepci�n de no implementado
\end{algorithmic}
\end{algorithm}

\section{Detalles de implementaci�n}

\section{Calculo de complejidad}

\subsection{Peores casos}

\section{Analisis experimental}

\subsection{Experiencias realizadas}

\subsection{Resultados}

\section{Discusi�n}
\chapter{Heuristicas Constructivas}

\section{Introducc�on}
Para enfrentar al problema, pensamos varias heuristicas constructivas distintas.

A continuaci�n se presentan las mismas, comentando las ideas sobre las que se basan, mostrando su aplicaci�n a un grafo de ejemplo y se muestra un pseudocodigo de cada una.

Posteriormente se realiza un estudio empirico de ellas comparando su desempe�o para minimizar el n�mero de cruces, asi como el tiempo que necesitan para poder ejecutarse. A partir de estas experiencias seleccionamos una heuristica para su implementaci�n definitiva.

Los pseudocodigos de este apartado son de caracter ilustrativo.

\section{Descripci�n de las heuristicas}

\subsection{Heuristica de inserci�n Greedy de nodos}
El primer enfoque que pensamos consiste en partir del dibujo original, es decir el que solo tiene los nodos cuyo orden esta
fijo e ir agregando los nuevos nodos con sus ejes, en la mejor posici�n en ese momento. Es decir, elegimos un nodo, y lo colocamos en la posici�n que genere menos cruces, teniendo en los nodos ya puestos.

Una vez hecho eso, se elige otro nodo (entre los que todavia no estan puestos) y se insertan en la misma manera.

La elecci�n se va a haciendo para cada partici�n, es decir se toma primero un nodo de los que tienen que ir en la primer partici�n y se inserta, luego se toma otro de la segunda. Cuando ya se insertaron todos los nodos de una, se continua con los de la otra

Por como procede esta heuristica, si hay ejes a agregar que tienen sus dos extremos en nodos que ya estaban en el dibujo, lo que
hacemos es ponerlos al principio, antes de hacer nada, porque asi obtenemos mas informaci�n para insertar a los nuevos nodos.

Hay varias formas de elegir a que nodo insertar. Nosotros consideramos tres formas distintas:
\begin{enumerate}
\item Escoger un nodo al azar entre los libre
\item Escoger el nodo de mayor grado hacia el dibujo armado (es decir el nodo que tenga mas adyacentes ya colocados)
\item Escoger el nodo de menor grado hacia el dibujo armado (el nodo que tenga menos adyacentes ya colocados)
\end{enumerate}

Vamos a aplicar la heuristica al siguiente grafo:

\begin{figure}[H]
    \centering
    \setcounter{subfigure}{0}
    \subfigure[]{
     \includegraphics[scale=0.25]{./figuras/constructivas/insercionGreedyRandom/posta.png}}
     \subfigure[]{
\includegraphics[scale=0.25]{./figuras/constructivas/insercionGreedyRandom/dibujo0.png} }
     \caption{dibujo �ptimo y dibujo de partida}       
      \label{fig:posta}
\end{figure} 

\begin{itemize}

\item Insercion por selecci�n random
\begin{figure}[H]
\centering
\setcounter{subfigure}{0}
\subfigure[]{
\includegraphics[scale=0.2]{./figuras/constructivas/insercionGreedyRandom/dibujo1.png}}
\subfigure[]{
\includegraphics[scale=0.2]{./figuras/constructivas/insercionGreedyRandom/dibujo2.png} }
\subfigure[]{
\includegraphics[scale=0.2]{./figuras/constructivas/insercionGreedyRandom/dibujo3.png}}
\subfigure[]{
\includegraphics[scale=0.2]{./figuras/constructivas/insercionGreedyRandom/dibujo4.png}}
\end{figure}

\item Inserci�n tomando mayor grado

\begin{figure}[H]
\centering
\setcounter{subfigure}{0}
\subfigure[]{
\includegraphics[scale=0.2]{./figuras/constructivas/insercionGreedyMayorGrado/dibujo1.png}}
\subfigure[]{
\includegraphics[scale=0.2]{./figuras/constructivas/insercionGreedyMayorGrado/dibujo2.png} }
\subfigure[]{
\includegraphics[scale=0.2]{./figuras/constructivas/insercionGreedyMayorGrado/dibujo3.png}}
\subfigure[]{
\includegraphics[scale=0.2]{./figuras/constructivas/insercionGreedyMayorGrado/dibujo4.png}}
\end{figure}

\item Insercion por menor grado
\begin{figure}[H]
\centering
\setcounter{subfigure}{0}
\subfigure[]{
\includegraphics[scale=0.2]{./figuras/constructivas/insercionGreedyMenorGrado/dibujo1.png}}
\subfigure[]{
\includegraphics[scale=0.2]{./figuras/constructivas/insercionGreedyMenorGrado/dibujo2.png} }
\subfigure[]{
\includegraphics[scale=0.2]{./figuras/constructivas/insercionGreedyMenorGrado/dibujo3.png}}
\subfigure[]{
\includegraphics[scale=0.2]{./figuras/constructivas/insercionGreedyMenorGrado/dibujo4.png}}
\end{figure}
\end{itemize}

%TODO: ver que esto se respete o sino cambiarlo :p
\subsubsection{Pseudocodigo}
\begin{algorithm}[H]
\caption{Propone un dibujo mediante la inserci�n golosa de nodos}
\begin{algorithmic}[1]
\PARAMS{un dibujo original, los nodos a colocar, y los ejes entre los nodos}
\WHILE{Queden nodos por poner}
\FOR{cada particion, si es posible}
\STATE nodo $\leftarrow$ elegir uno entre los nodos a poner
\STATE sacar al nodo de entre los nodos a poner
\STATE colocar al nodo en la primer posicion de su partici�n
\STATE cruces $\leftarrow$ cuantos cruces se agregan
\STATE mejorCruces $\leftarrow$ cruces
\STATE crucesPreSwap $\leftarrow$ cruces entre el nodo y el nodo siguiente en la particion
\STATE mejorPos = 0
\WHILE{No revise todas las posiciones}
\STATE mover al nodo a la proxima posici�n
\STATE crucesPreSwap $\leftarrow$ cruces entre el nodo y el nodo siguiente en la particion
\STATE cruces $\leftarrow$ cruces - crucesPreSwap + crucesPostSwap
\IF{ cruces $<$ mejorCruces}
\STATE mejorCruces $\leftarrow$ cruces
\STATE mejorPos $\leftarrow$ la posicion donde esta ahora
\ENDIF
\STATE crucesPreSwap $\leftarrow$ crucesPostSwap
\ENDWHILE
\STATE poner al nodo finalmente en mejorPos
\ENDFOR
\ENDWHILE
\end{algorithmic}
\end{algorithm} 

%TODO: capaz al pedo
%\subsubsection{Calculo de complejidad}
%Para determinar la complejidad de este algoritmo, as� como para las futuras heuristicas, utilizaremos el modelo uniforme, ya que consideramos el la parte central del problema no esta en el costo que puedan tener las operaciones matematicas que se realicen sino la cantidad de nodos y ejes del dibujo.
%
%De acuerdo con esto, veamos que complejidad tiene el algoritmo, para esto asumiremos que m $\leq v_1$ $\wedge$ m $\leq v_2$, con $v_i$ la cantidad de nodos de la partici�n i luego de sacar a los nodos sin ejes, y $m$ la cantidad de ejes del grafo; ya que como dijimos anteriormente, los nodos sin ejes se pueden sacar antes de procesar. Tenemos, entonces un costo por sacar a los nodos sin ejes, esto se puede hacer en $O( V_1 +  v_2)$ donde $V_i$, es la cantidad original de nodos, antes de sacar los que ten�an grado nulo. Para hacerlo simplemente recorremos todos los nodos, preguntando cuantos adyacentes tiene cada uno. Notemos que $v_i \leq V_i$.
%
%Entonces, en el algoritmo, si bien no esta en el pseudocodigo, se crea un indice con las posiciones que ocupan los nodos ya puestos en cada partici�n, lo cual sirve para poder hacer mas r�pidamente los calculos de cruces. Armar este indice, que no es mas que un arreglo de $v_1 +  v_2$ posiciones, tiene costo $O(p_1 + p_2)$ con $p_i$ los nodos ya puestos.
%
%Luego tenemos un ciclo que se ejecuta mientras queden nodos por colocar, es decir $IV_1 + IV_2$ iteraciones, con $IV_i$ la cantidad de nodos a agregar en la partici�n i.
%
%En este ciclo se elige un candidato para colocar. Digamos por ahora que esta elecci�n tiene costo C.
%
%Luego se saca al candidato, lo cual tiene costo $O(v_i)$ porque los candidatos los tenemos en una lista y podriamos tener que sacar un candidato de la mitad de la lista %FIXME: completenme
%
%A continuaci�n, insetamos al nodo en su partici�n, y contamos los cruces que se agregan, que como dijimos se puede hacer en $O(  p_1 +   p_2 + m)$.
%
%A partir de aqu�, lo que hacemos es ir intercambiando al nodo con su inmediato siguiente, viendo cuantos cruces se producen por el intercambio, y cuantos dejan de existir. Esto ultimo lo logramos con $O(  p_opuesta +   m_x +   m _y)$ donde $p_opuesta$ son los nodos de la partici�n opuesta a la de x e y y, $m_x$ y $m_y$ son los ejes de los nodos que se van a swapear.
%Como $p_opuesta \subseteq v_opuesta$ y ademas $m_x\subseteq m \wedge m_y \subseteq m$ podemos decir que cada iteraci�n es, acotando, $O(m+v_opuesta)$. Como hacemos $  p_i$ iteraciones,  y $p_i \subseteq v_i$, resulta que todo el ciclo es $O(  m*  v_i +   v_i*   v_opuesta)$ 
%
%Una vez que decidimos en que posici�n lo vamos a insertar (es decir en que posici�n el n�mero de crueces fue m�nimo) esta inserci�n tiene $O(  p_i)$. Al igual que actualizar el indice (solo se actualiza la parte correspondiente a la partici�n modificada).
%
%Entonces tenemos un ciclo que itera $  IV_1 +   IV_2$, y sumando los costos internos al ciclo tenemos que cada iteraci�n tiene costo:  
%
%$O(  m*  v_i +   v_i*   v_opuesta +   p_1 +   p_2 +   m + C +   v_i)$
%
%Usando que $  v_i*   v_opuesta = v_1*v_2$ y que $p_i \subseteq v_i$, lo que obtenmos es:
%
%$O(  m*  v_i +   v_1*v_2 +   v_1 +   v_2 +   m + C +   v_i)$
%
%podemos ademas acotar $  v_i$ por $  v_{max}$, resulta que nos queda:
%
%$O(  m*  v_{max} +   v_{max}^2 +   v_{max} +   m + C +   v_{max})$
%
%como este costo es el de una iteraci�n, nos queda que el orden del ciclo completo es:
%
%$O((  IV_1 +   IV_2)(  m*  v_{max} +   v_{max}^2 +   v_{max} +   m + C +   v_{max}))$
%
%pero $  IV_1 \leq   IV_{max}$ y de la misma manera vale lo mismo para $IV_2$, luego tenemos que el orden es:
%
%$O(   IV_{max}(  m*  v_{max} +   v_{max}^2 +   v_{max} +   m + C +   v_{max}))$
%
%si bien, $  IV_{max}$ se puede acotar por $  v_{max}$, preferimos dejarlo de esta manera, ya que evidencia mejor como el aumento de la cantidad de nodos a poner influye mas que la cantidad de nodos fijos.
%
%Por otro lado no hay que olvidarse del costo de filtrar los nodos de grado 0 y volverlos a colocar, lo cual tiene orden
%
%$O(   V_{max})$, entonces el orden total es  
%
%$O(   V_{max} +   IV_{max}(  m*  v_{max} +   v_{max}^2 +   v_{max} +   m + C +   v_{max}))$
%
%Con respecto al costo C, en el caso de la elecci�n random, el costo es O(1) y en la elecci�n por grado m�ximo o m�nimo, esta es $O(  v_{max})$.
%
%Para este algoritmo, el peor caso se da cuando hay muchos nodos para agregar y el grafo es muy denso. En particular si tenemos que no hay nodos fijos y el grafo es completo, el costo nos queda $O( V_{max}^4)$ porque $IV_i = v_i = V_i$ y ademas es del orden de $V_i^2$. 
\subsection{Heuristica de insercion Greedy por ejes}
Nuevamente partimos del dibujo original, pero esta vez vamos agregando ejes. Es decir tomamos un eje de los que vienen en el nuevo dibujo y lo agregamos poniendo a sus nodos en la posici�n que minimize el n�mero de cruces. Si tomamos un eje que une dos nodos que no fueron puestos aun, se agregan ambos nodos y se prueban las distintas combinaciones para minimizar los cruces. Si alguno (o ambos extremos) ya estaban puestos, se sacan ambos y se reubican. 

Esta reubicaci�n tiene mas informaci�n que la primera ubicaci�n, ya que por lo menos ambos tienen un eje ya colocado, por lo que podr'ia mejorar incluso la cantidad de cruces que habia antes de agregar el eje, cosa que con la heuristica anterior no ocurre: en la heuristica de inserci�n de nodos, cada vez que se colocaba un nodo el n�mero de cruces aumentaba o permanec�a igual; pero nunca puede bajar.

Por otro lado, si bien parecer�a que puede lograr mejores resultados que la otra heuristica, hay que tener en cuenta que va a resultar mas costosa, ya que para cada eje hay que recorrer toda la primer partici�n, y para cada posici�n de esta, se recorre la segunda, viendo cuantos cruces se originan. Por esta raz�n, es importante analizar no solo el desempe�o de esta heuristica en cuanto a reducir el n�mero de cruces, sino tambi�n en cuanto al tiempo que demora, ya que podr�a ser considerablemente mas alto que el de las dem'as heuristicas.

Si aplicamos la heuristica para \ref{fig:posta}, obtenemos lo siguiente:

\begin{figure}[H]
\centering
\setcounter{subfigure}{0}
\subfigure[]{
\includegraphics[scale=0.2]{./figuras/constructivas/insercionEjes/dibujo1.png}}
\subfigure[]{
\includegraphics[scale=0.2]{./figuras/constructivas/insercionEjes/dibujo2.png} }
\subfigure[]{
\includegraphics[scale=0.2]{./figuras/constructivas/insercionEjes/dibujo3.png}}
\subfigure[]{
\includegraphics[scale=0.2]{./figuras/constructivas/insercionEjes/dibujo4.png}}
\end{figure}

Notemos como en el ultimo paso, agrega el eje (3,8) mueve al 8 de la posici�n que le habia asignado antes, de modo de reducir
la cantidad de cruces. Por otro lado notemos que si bien logro la soluci�n �ptima, esto se debi� a que cuando podia elegir donde poner a los nodos, los puso abajo, si se hubiese elegido ponerlos arriba (opci�n valida, dado que genera la misma cantidad de cruces, es decir 0) el resultado hubiera sido distinto.

\subsubsection{Pseudocodigo}
\begin{algorithm}[H]
\caption{Propone un dibujo mediante la inserci�n golosa de ejes}
\begin{algorithmic}[1]
\STATE ejesPuestos $\leftarrow$ los ejes del dibujo original
\STATE puesto$[v_i]$ $\leftarrow$ si $v_i$ estaba en el dibujo original entonces True sino False
\FOR{cada eje(x,y) a agregar}
\IF{ya puse a x}
   \STATE sacarlo
\ELSE
   \STATE marcarlo como ya puesto
\ENDIF
\IF{ya puse a y}
   \STATE sacarlo
\ELSE
   \STATE marcarlo como ya puesto
\ENDIF
\STATE agregar el eje a los ejes Puestos 
\STATE agregar a x a la lista de adyacencia de y
\STATE agregar a y a la lista de adyacencia de x
\STATE calcular los rangos en los cuales puedo mover a $x$ y a $y$ \COMMENT{si alguno estaba en el dibujo original, hay que respetar el orden relativo}
\STATE insertar a x en su primer posici�n valida
\STATE insertar a y en su primer posici�n valida
\STATE mejoresCruces $\leftarrow$ los cruces por ponerlos en esta posici�n
\STATE mejorPosici�n $\leftarrow$ posici�n actual
\FOR{cada posicion valida para x}
\FOR{cada posici�n valida para y}
\STATE contar los cruces por dejarlos en esa posici�n
\IF{generan menos cruces que mejoresCruces}
   \STATE mejoresCruces $\leftarrow$ cruces por tenerlos en esta posicion
   \STATE mejorPosici�n $\leftarrow$ posicion actual
\ENDIF
\STATE mover y a su proxima posici�n
\ENDFOR
\STATE mover a x a su proxima posici�n
\STATE mover a y a su primer posici�n valida
\ENDFOR
\STATE mover a $x$ y a $y$ a la mejorPosicion
\ENDFOR
\end{algorithmic}
\end{algorithm} 

\subsection{Heuristica de construcci�n por mediana}
La idea de esta heuristica es buscar que ning�n nodo este ``demasiado�� lejos de sus adyacentes. Para lograr esto utilizamos la mediana de las posiciones de sus adyacentes.

El procedimiento es el siguiente, en un principio se comienza con solamente los nodos que ya estaban en el dibujo y sus ejes.

Tomamos entonces al nodo de mayor grado (con respecto a los nodos que ya estan puestos), calculamos la mediana de las posiciones de sus adyacentes y una vez que la obtenemos, probamos insertar al nodo en la posici�n de su mediana, o en la posici�n de su mediana mas o menos uno. Elegiendo de las tres la que genere menos cruces. En el caso en que la mediana no sea un indice valido, porque una partici�n tiene menos nodos y el valor de la mediana (que se calcula a partir de la posici�n de los nodos de la mas grande) supera a la cantidad de nodos de la misma, la truncamos. Repetimos el procedimiento hasta que esten puestos todos los nodos.

\begin{figure}[H]
\centering
\setcounter{subfigure}{0}
\includegraphics[scale=0.25]{./figuras/constructivas/medianaTruncada.png}
\caption{ejemplo de inserci�n por mediana con truncamiento}
\end{figure}

Al igual que en la heuristica de insercci�n de nodos, si habia ejes a agregar entre los nodos que ya estaban, estos se agregan al inicio para dar mas informaci�n a las posteriores inserciones.

Si aplicamos la heuristica a \ref{fig:posta} lo que se obtiene es, en este ejemplo, lo mismo que en la inserci�n greedy de nodos. Ya que siempre se agregan nodos que tienen un adyacente, el cual esta ubicado al final de la partici�n.

Esta heuristica utiliza un criterio greedy indirecto, es decir, las otras heuristicas son greedys en cuanto al n�mero de cruces que se originan por cada insercion, en cambio esta es greedy en la distancia a la que queda cada nodo de sus adyacentes. Es una heuristica similar a la del baricentro, y que en el caso del problema de dibujo de grafos bipartitos sin la caracteristica de ser incrementales se muestra como una heuristica buena %TODO citar paper :p

Luego de ubicar a todos los nodos, se hace una pasada en cada partici�n intercambiando nodos en posiciones consecutivas si generan menos cruces, para de esta manera intentar reducir el efecto que puede tener el truncamiento de la mediana cuando todavia no estan todos los nodos puestos. Es decir, si un nodo se inserta en una partici�n con pocos nodos, y su mediana da muy alta, esta se trunca. Entonces queda con la misma mediana que otros nodos que tenian su mediana mas chica. Entonces en un intento de paliar esta situcaci�n es que se intenta intercambiar posiciones consecutivas que generen menos cruces.

\subsubsection{Pseudocodigo}
\begin{algorithm}[H]
\caption{Propone un dibujo mediante la inserci�n por la mediana de los adyacentes}
\begin{algorithmic}[1]
\WHILE{queden nodos sin poner}
\STATE elegir un nodo de grado maximo con respecto a lo que ya esta puesto
\STATE calcular la mediana de las posiciones de sus adyacentes
\IF{mediana $>$ tama�o actual de la partici�n}
\STATE mediana $\leftarrow$ tama�o de la partici�n
\ENDIF
\FOR{ cada i = mediana-1,mediana,mediana+1}
\IF{es una posici�n valida}
\STATE contar los cruces por ponerlo en esa posici�n
\IF{lo inserte por primera vez o me genera menos cruces que la mejor posicion}
\STATE mejor posicion $\leftarrow$ posici�nActual
\ENDIF
\ENDIF
\ENDFOR
\STATE poner al nodo en la mejor posici�n de las 3
\ENDWHILE
\end{algorithmic}
\end{algorithm} 

\section{Comparaci�n de las heuristicas constructivas}
A fin de decidir cual o cuales de estas heuristicas se comporta mejor, decidimos hacer primero una implementaci�n en python, lenguaje que nos resulta mas comodo que C++ o Java. Utilizando estas implementaciones, aplicar las heuristicas y comparar los resultados, teniendo en cuenta no solo la cantidad de cruces, sino tambi�n el tiempo que le toma a cada una proponer un dibujo.

Lo que esperamos es que la heuristica de insercci�n de ejes de mejores resultados por el hecho de que reinserta nodos, por lo cual tiene varias oportunidades para fijarlos, por lo cual podr�a corregir errores cometidos por insertar cuando todavia habia pocos nodos puestos.

Sin embargo, creemos que este metodo puede ser considerablemente mas lento que los demas. Por un lado porque itera tantas veces como ejes se agreguen, lo cual podria $O(n^2)$ con n la cantidad de nodos. Ademas cada una de estas iteraciones requiere de $O(n^2)$ intercambio de nodos. Si a esto le sumamos el costo de contar los cruces vemos que el orden es bastante elevado.

Por otro lado no estamos seguros a priori de que resultados puede dar la heuristica de la mediana, ya que calcular la mediana cuando todavia no esta fija ninguna de las particiones podr�a no brindar suficiente informaci�n. Sin embargo, creemos que va a ser el m�todo mas rapido, ya que para cada nodo solo hace a lo sumo tres intentos de inserci�n.

Finalmente con respecto a la inserci�n golosa de los nodos, creemos que su costo ser� menor que el de la inserci�n de ejes, pero sus resultados podrian no ser tan buenos.

Para probarlos corrimos los siguientes tests:
\begin{enumerate}
\item Comparaci�n de Heuristicas de inserci�n golosa de nodos:
Primero comparamos a las diferentes formas de elegir al nodo candidato, para observar si alguna de las formas de hacerlo, se desempe�aba mejor.

\item Comparaci�n entre Heuristicas
\begin{enumerate}
\item n nodos en cada partici�n con n creciente. Cantidad de ejes = $\frac{n^2}{2}$. Porcentaje de nodos nuevos: 60\%
\item n nodos en cada partici�n con n creciente. Cantidad de ejes = $\frac{n^2}{2}$. Porcentaje de nodos nuevos: 40\%
\item n nodos en cada partici�n con n creciente. Cantidad de ejes = 3n. Porcentaje de nodos nuevos: 60\%
\item n nodos en cada partici�n con n creciente. Cantidad de ejes = 3n. Porcentaje de nodos nuevos: 40\%
%\item n = 30. Cantidad de ejes creciente. Porcentaje de nodos nuevos: 60%
\item n = 30. Cantidad de ejes creciente. Porcentaje de nodos nuevos: 40\%
\end{enumerate}
\end{enumerate}

En cada uno de ellos, se midi� la cantidad de cruces y el tiempo utilizado para lograr el dibujo.

Si bien consideramos que los tiempos de ejecuci�n en un lenguaje interpretado como lo es Python, son por lo general mayores que los tiempos de ejecuci�n en C++, creemos que son igualmente validos para permitirnos observar una tendencia general en el comportamiento de las heuristicas. Por otro lado, dado que implementar en este lenguaje nos resulta mucho mas sencillo, consideramos que vale la pena probar a las tres heuristicas en vez de simplemente proponer una unica para implementar en C++.

\subsection{Criterios de selecci�n de nodos para la heuristica de inserci�n de nodos}
Las pruebas que realizamos consistieron en aplicar la heuristica de insercion de nodos a grafos aleatorios variando la cantidad de nodos en cada partici�n.

En la primer experiencia utilizamos grafos con $m=2*n$ y un $40\%$ de nodos fijos (En adelante n es la cantidad de nodos de cada partici�n).

En la segunda experiencia la cantidad de ejes estaba dada por $m=\frac{n^2}{2}$ y el porcentaje de nodos fijos fue tambi�n del $40\%$.

La idea fue observar si alguno de los criterios para elegir que nodo insertar en cada paso lograba un mejor desempe�o.

Los resultados se encuantran en los graficos 

\begin{figure}[H]
\centering
\setcounter{subfigure}{0}
\subfigure[]{
\includegraphics[scale=0.6]{./graficos/comparacionInsercionNodos/exp2.png}}
\subfigure[]{
\includegraphics[scale=0.6]{./graficos/comparacionInsercionNodos/exp1.png}} 
\end{figure}

De estas experiencias, notamos que la cantidad de cruces encontrada por los tres metodos, es relativamente similar. Sin embargo lo que notamos es que el criterio de mayor grado parecer�a comportarse ligeramente mejor que los otros dos. Para nosotros tiene sentido que esto sea as�, ya que si se utiliza el nodo de mayor grado con respecto a lo que ya esta puesto, ese nodo tiene mas informaci�n (mas adyacentes puestos) por lo que puede ubicarse mejor. Claro esta que esto podr�a fallar, sin embargo a partir de esta idea, mas lo que se observa en las pruebas, decidimos utilizar al nodo de mayor grado para la heuristica de inserci�n de nodos.

\subsection{Comparaci�n de heuristicas constructivas}
Los resultados de las pruebas realizadas son los siguientes:

\begin{figure}[H]
\centering
\setcounter{subfigure}{0}
\subfigure[Cantidad de cruces producidos en funcion de n]{
\includegraphics[scale=0.61]{./graficos/comparacionConstructivas/cruces1.png}}
\setcounter{subfigure}{1}
\subfigure[Tiempo en segundos en funci�n de n]{
\includegraphics[scale=0.61]{./graficos/comparacionConstructivas/tiempos1.png} }
\caption{n nodos en cada partici�n con n creciente. Cantidad de ejes = $\frac{n^2}{2}$. Porcentaje de nodos nuevos: 60\%}
\end{figure}

\begin{figure}[H]
\centering
\setcounter{subfigure}{0}
\subfigure[Cantidad de cruces producidos en funcion de n]{
\includegraphics[scale=0.61]{./graficos/comparacionConstructivas/cruces2.png}}
\setcounter{subfigure}{1}
\subfigure[Tiempo en segundos en funci�n de n]{
\includegraphics[scale=0.61]{./graficos/comparacionConstructivas/tiempos2.png} }
\caption{{ n nodos en cada partici�n con n creciente. Cantidad de ejes = $\frac{n^2}{2}$. Porcentaje de nodos nuevos: 40\%}}
\end{figure}

\begin{figure}[H]
\centering
\setcounter{subfigure}{0}
\subfigure[Cantidad de cruces producidos en funcion de n]{
\includegraphics[scale=0.61]{./graficos/comparacionConstructivas/cruces3.png}}
\setcounter{subfigure}{1}
\subfigure[Tiempo en segundos en funci�n de n]{
\includegraphics[scale=0.61]{./graficos/comparacionConstructivas/tiempos3.png} }
\caption{ n nodos en cada partici�n con n creciente. Cantidad de ejes = 3n. Porcentaje de nodos nuevos: 60\%}
\end{figure}

\begin{figure}[H]
\centering
\setcounter{subfigure}{0}
\subfigure[Cantidad de cruces producidos en funcion de n]{
\includegraphics[scale=0.61]{./graficos/comparacionConstructivas/cruces4.png}}
\setcounter{subfigure}{1}
\subfigure[Tiempo en segundos en funci�n de n]{
\includegraphics[scale=0.61]{./graficos/comparacionConstructivas/tiempos4.png} }
\caption{ n nodos en cada partici�n con n creciente. Cantidad de ejes = 3n. Porcentaje de nodos nuevos: 40\%}
\end{figure}

\begin{figure}[H]
\centering
\subfigure[Cantidad de cruces producidos en funcion de m]{
\includegraphics[scale=0.61]{./graficos/comparacionConstructivas/cruces5.png}}
\subfigure[Tiempo en segundos en funci�n de m]{
\includegraphics[scale=0.61]{./graficos/comparacionConstructivas/tiempos5.png} }
\caption{ n = 30. Cantidad de ejes creciente. Porcentaje de nodos nuevos: 40\%}
\setcounter{subfigure}{0}
\end{figure}

\section{An�lisis de los resultados}
Al observar los gr'aficos de las experiencias lo primero que salta a la vista es que el tiempo de ejecuci�n de la heur'istica de inserci�n de ejes es mucho mas grande que el de las demas. Esta situaci�n que se hace mas notoria en grafos densos hace que su uso no sea recomendable, mas si tenemos en cuenta a partir de las demas experiencias, que los resultados que obtiene no son significativamente mejores que el del resto de las heuristicas. Por ejemplo, en la experiencia 5 vemos como para un grafo con 30 nodos y 799 ejes la diferencia entre la inserci�n de nodos y la inserci�n de ejes de 442 cruces a favor de la isnserci�on de ejes (142131 contra 141689), lo cual representa una diferencia del 0.3120 \%, pero en cuanto a tiempos para la misma instancia, insercion de nodos demor� 0.5160 segundos contra 13.6870 que demor� la inserci�n de ejes, es decir 26.5252 veces mas.

Por esta raz�n, si bien en general vemos que da resultados relativamente buenos, decidimos descartarla.

Por otro lado, la heuristica de la mediana se muestra como la mas rapida, suponemos que debido a que la cantidad de veces que necesita contar cruces, ya sea entre dos nodos o en todo el grafo, es mucho menor que la de las otras. Sin embargo, en cuanto a la cantidad de cruces, suele dar peores resultados que las otras dos. Es por esto, que decimos descartar tambi�n esta heuristica tambi�n.

De esta manera elegimos implementar en C++ la heuristica de inserci�n de nodos, ya que consideramos que de las tres alternativas planteadas es la que brinda resultados bastante buenos (comparada con las demas heuristicas planteadas) en tiempos razonables. 


\section{Detalles de implementaci�n de la heuristica constructiva}

%darle rigurosidad al pseudocodigo correspondiente para poder usarlo aca
\section{Calculo de complejidad}

\section{Analisis experimental}
\subsection{Casos borde}

\subsection{Casos generales}

\section{Discusi�n}
\chapter{Busqueda Local}

\section{Introducci�n}
De manera analoga a lo que hicimos para las heuristicas constructivas, plantemos diferentes heuristicas de busqueda local. 

A continuaci�n comentaremos como proceden dichas heuristicas, y posteriormente realizaremos diversas experiencias para poder decidir a partir de estas cual utilizaremos en el GRASP.

\section{Descripci�n de las heuristicas}

\subsection{Busqueda local por reinserci�n de nodos}
Esta m�todo de busqueda local procede tomando cada nodo, sacandolo del dibujo y reubicandolo en la mejor posici�n, en el sentido de que se generan menos cruces. Este procedimiento se repite para cada nodo del dibujo.

Cada paso de la busqueda local consiste entonces en reinsertar cada nodo del dibujo una vez. Consideramos que estamos en un m�nimo local si la cantidad de cruces antes y despu�s de un paso es la misma.

En el caso de los nodos cuyo orden relativo debe ser respetado, la reinserci�n se realiza entre posiciones posibles que no violen dicho invariante

Veamos el siguiente ejemplo de aplicaci�n de la busqueda local por reinserci�n (para simplificar no se consideraron nodos fijos):

\begin{figure}[H]
    \centering
    \setcounter{subfigure}{0}
    \subfigure[Dibujo a mejorar (4 cruces)]{
     \includegraphics[scale=0.3]{./figuras/BusquedaLocal/reinsercion.png}}     
     \setcounter{subfigure}{1}
     \subfigure[Buscamos a donde reinsertae al nodo A, delante de D logramos minimizar los cruces]{
     \includegraphics[scale=0.3]{./figuras/BusquedaLocal/reinsercion1.png}}     
     \setcounter{subfigure}{2}
     \subfigure[Movemos al nodo A, no podemos mover a nadie mas de esta partici�n de modo de bajar el n�mero de cruces, por lo cual, pasamos a la siguiente partici�n. Moviendo a 4 no logramos nada, por lo que buscamos mover a 3(1 cruce)]{
     \includegraphics[scale=0.3]{./figuras/BusquedaLocal/reinsercion2.png}}     
     \setcounter{subfigure}{3}
     \subfigure[Movimos a 3, y ya no queda ninguna mejora por hacer (0 cruces)]{
     \includegraphics[scale=0.3]{./figuras/BusquedaLocal/reinsercion3.png}}     
\end{figure} 

\subsubsection{Pseudocodigo}
\begin{algorithm}[H]
\caption{Intenta mejorar un dibujo mediante la reinserci�n golosa de nodos}
\begin{algorithmic}[1]
\FOR{cada nodo del dibujo}
\STATE sacar al nodo del mismo
\STATE obtener las posiciones donde es posible insertarlo
\STATE mejoresCruces $\leftarrow$ cruces por ponerlo en la primer posici�n posible
\STATE mejorPosici�n $\leftarrow$ primer posici�n
\FOR{cada posici�n donde se puede poner al nodo}
\STATE crucesActuales $\leftarrow$ cruces por ponerlo en dicha posici�n
\IF{crucesActuales $<$ mejoresCruces}
\STATE mejoresCruces $\leftarrow$ crucesActuales
\STATE mejorPosici�n $\leftarrow$ posici�n actual
\ENDIF
\ENDFOR
\STATE poner al nodo en la mejor posici�n
\ENDFOR
\end{algorithmic}
\end{algorithm} 

\subsection{Busqueda local por intercambio goloso de nodos}
Esta heuristica contempla como soluciones vecinas de un dibujo a aquellas que se pueden obtener por un intercambio v�lido entre dos nodos del dibujo.

\begin{figure}[H]
    \centering
     \includegraphics[scale=0.5]{./figuras/BusquedaLocal/vecindad.png}
\end{figure}

Primero se considera la vecindad, consistente en todo posible intercambio de dos nodos (siempre que dicho intercambio no viole el orden relativo de los nodos originales) y luego prueba cual de todos esos intercambios reporta mayor beneficio, es decir reduce mas el n�mero de cruces. Una vez encontrado dicho par, nos movemos a la soluci�n vecina realizando el intercambio de dichos nodos. Al hacerlo terminamos un paso de la busqueda local.

El procedimiento se repite hasta que ning�n intercambio genere una reducci�n en el n�mero de cruces. En cuyo caso decimos que alcanzamos un m�nimo local.

\subsubsection{Pseudocodigo}
\begin{algorithm}[H]
\caption{Intenta mejorar un dibujo mediante intercambio goloso de nodos}
\begin{algorithmic}[1]
\STATE vecindad = \{(x,y) por cada x en alguna particion e y de la misma partici�n, si es v�lido intercambiar x por y\}
\STATE mejorIntercambio $\leftarrow$ ninguno
\STATE crucesPorIntercambio $\leftarrow$ cantidad de cruces del dibujo
\FOR{(x,y) en vecindad}
\STATE crucesVecino $\leftarrow$ cantidad de cruces al intercambiar x e y
\IF{crucesVecino $<$ crucesPorIntercambio}
\STATE mejorIntercambio $\leftarrow$ (x,y)
\STATE crucesPorIntercambio $\leftarrow$ cruces al intercambiar x e y
\ENDIF
\ENDFOR
\IF{mejorIntercambio $\neq$ ninguno}
\STATE realizar el intercambio
\ENDIF
\end{algorithmic}
\end{algorithm} 

\subsection{Busqueda local por inserci�n por mediana}
Una de las heuristicas constructivas que planteamos es la inserci�n de nodos por mediana. Esta heuristica no funcion� bien como esperabamos, ya que si bien era r�pida, generaba mas cruces que las otras heuristicas golosas. Nuestra idea entonces es aplicar el concepto de la mediana, pero como busqueda local.

En este contexto como todos los nodos estan puestos, cada nodo tiene ahora la informaci�n de todos sus adyacentes, es por esta raz�n que creemos que podria funcionar bien el metodo como busqueda local.

Entonces la idea es muy similar a la inserci�n por mediana: tomamos cada nodo de una partici�n y tratamos de moverlo a la posici�n correspondiente a la mediana de las posiciones de sus adyacentes, o la mediana mas o menos uno. Si al moverlo se reducen los cruces lo hacemos. Si esto no ocurre, se lo deja donde esta. A diferencia de la heuristica constructiva, en esta los nodos que estaban en el dibujo inicial tambi�n se intentan ubicar segun sus medianas siempre que esto no rompa el orden relativo que deben guardar.

 Una vez hecho esto para todos los nodos, lo que hacemos es tratar de intercambiar adyacentes, con el objetivo de reducir el n�mero de cruces.

La busqueda termina cuando no es posible reducir el n�mero de cruces ya sea ubicando en la posici�n de la mediana o por intercambio de pares.

\subsubsection{Pseudocodigo}
\begin{algorithm}[H]
\caption{Intenta mejorar un dibujo inserci�n por mediana}
\begin{algorithmic}[1]
\FOR{ cada nodo del dibujo}
\STATE calcular la mediana de las posiciones de los adyacentes al nodo
\STATE mejorPos $\leftarrow$ posicionActual
\STATE mejoreCruces $\leftarrow$ cruces en el dibujo
\FOR{ posicion = mediana -1, mediana, mediana + 1}
\IF{ se puede insertar en esa posici�n  y baja el n�mero de cruces en el dibujo}
\STATE mejorPos $\leftarrow$ posicion
\STATE mejoresCruces $\leftarrow$ cruces en el dibujo al poner al nodo en posicion
\ENDIF
\ENDFOR
\STATE poner al nodo en mejorPos
\ENDFOR
\end{algorithmic}
\end{algorithm} 

\chapter{GRASP}

\section{Modificaciones a la heuristica constructiva}
Para poder aplicar nuestra heuristica constructiva a un procedimiento GRASP, fue necesario introducir algun factor de aleatoriedad a la misma.

Nosotros consideramos dos formas de hacerlo:
\begin{enumerate}
\item Modificar el criterio de elecci�n del nodo candidato en la inserci�n:
Se considera un valor $\alpha \in [0,1]$, de modo que en cada paso no se selecciona el de grado m�ximo, sino que se selecciona un v tal que $d(v) \geq \alpha*d_{max}$. Si $\alpha = 1$, la elecci�n no es aleatoria, en cambio si $\alpha = 0$, se escoge un candidato totalmente al azar. En general, en (0,1), un $\alpha$ mas grande implica una lista restringida de candidatos mas peque�a.

\item Modificar el criterio de elecci�n de la posici�n:
Nuestra heuristica constructiva frente a un ``empate'' de posiciones, es decir, para un nodo dado, hay dos o mas posiciones que generan la misma cantidad de cruces, lo que hace es quedarse con la primera visitada. 

\begin{figure}[H]
\centering
\setcounter{subfigure}{0}
\subfigure[]{
\includegraphics[scale=0.2]{./figuras/grasp/empate1.png}}
\setcounter{subfigure}{1}
\subfigure[]{
\includegraphics[scale=0.2]{./figuras/grasp/empate2.png}}
\subfigure[]{
\includegraphics[scale=0.2]{./figuras/grasp/empate3.png}}
\caption{Cualquiera de las 3 posiciones para v es a priori tan buena como las otras}
\end{figure}

Entonces podemos modificar esto, para que si hay empate eliga alguna de todas estas posiciones al azar.

\end{enumerate}
Posteriormente realizaremos, experiencias con el fin de determinar si estas modificaciones son utilies, y ademas con el fin de determinar que valor de $\alpha$ debe usarse.

\section{Determinaci�n de los parametros}
Para nuestro grasp deb�amos fijar tres parametros:
\begin{enumerate}
\item criterio de parada
\item alfa (que determina el tama�o de la lista restringida de candidatos)
\item posici�n aleatoria, que determina si frente a un empate de posiciones nos quedamos con la primera encontrada o con alguna al azar.
\end{enumerate}
Para cada parametro se propusieron distintos valores alternativos, y luego realizamos experiencias para determinar cual de las distintas combinaciones daba un mejor resultado en funci�n de la cantidad de cruces obtenida y el tiempo requerido.
\subsection{Criterios de parada}
Consideramos que el criterio de parada debia tener en cuenta la cantidad de nodos que posee el grafo, ya que esta cantidad influye en la cantidad posible de configuraciones y por ende en la cantidad de mejoras que se pueden hacer. Por ejemplo como vimos en la busqueda local, en general necesitaban mas iteraciones para mejorar una soluci�n. Por esta raz�n el primer criterio que proponemos es el maximo de los tama�os de las particiones, criterio que nos da un una cantidad de iteraciones lineal en la cantidad de nodos.

El otro criterio que planteamos varia su cantidad de iteraciones de una manera adaptativa. Se toma como valor maximo en el numero de iteraciones la cantidad de nodos del grafo. Ahora si en un iteracion no se produce una mejora, se disminuye en 1 la cantidad de iteraciones. Si en cambio se produce una mejor, la cantidad maxima de iteraciones se divide por 2. 
Este criterio utiliza como el anterior, la idea de que mas nodos implica mas trabajo para mejorar, pero por otro lado agrega la idea de que no es posible mejorar indefinidamete y que si se mejoro mucho es menos probable que se siga mejorando.

En el peor caso, si no mejora nunca, este criterio hace que el grasp genere mas iteraciones que el criterio anterior, pero en el mejor caso hace una cantidad logaritmica de iteraciones.
% deberian ir juntos
\subsection{Tama�o de la lista de candidatos}
Para determinar el tama�o de la lista de candidatos propusimos tambi�n dos opciones:
\begin{itemize}
\item Tomar un alfa fijo = 0.75: La idea es que un valor bajo de alfa genera muchas soluciones, entre las cuales puede haber muchas malas, mientras que un alfa demasiado grande limita mucho la variedad de las soluciones generadas. Por esa razon nos parece que alfa = 0.75 podria ser un valor razonable.
\item Tomar un alfa adaptativo: En este caso, se parte de un alfa alto, 0.95, y en cada iteracion, si no se produce mejora, lo que se hace es disminuir el valor de alfa. De esta manera, la lista de candidatos comienza siendo peque�a, con la esperanza de lograr buenos resultados, y a medida que no se mejora, se da lugar a soluciones mas variadas
\end{itemize}
\subsection{Posicion aleatoria}
En este caso, se consideraron las dos alternativas: tomar posicion aleatoria o tomar la primer posici�n.

\subsection{Experimentos}
Con el fin de observar si alguna configuraci�n de los parametros, se comportaba mejor que las demas, decidimos aplicar cada posible configuraci�n a distintas instancias del problema.
Decidimos identificar a cada combinaci�n mediante un n�mero, lo cual hicimos de la siguiente manera:
\begin{enumerate}
\item alfa 0.75, primera posici�n, parada por maximo de partici�n
\item alfa 0.75, primera posici�n, parada adaptativa
\item alfa 0.75, posici�n aleatoria, parada por maximo de partici�n
\item alfa 0.75, posici�n aleatoria, parada adaptativa
\item alfa adaptativa, primera posici�n, parada por maximo de partici�n
\item alfa adaptativa, primera posici�n, parada adaptativa
\item alfa adaptativa, posici�n aleatoria, parada por maximo de partici�n
\item alfa adaptativa, posici�n aleatoria, parada adaptativa
\end{enumerate}
Para comparar, decidimos medir el tiempo que requer�a cada heuristica y ademas considerar cuanto lograban disminuir la cantidad de cruces, con respecto a la busqueda local. Es decir dada la soluci�n inicial con la que comienza el grasp, que tanto logra mejorarla.

Realizamos las siguientes experiencias:
\begin{itemize}
\item Aplicar la heuristica a grafos densos con entre 30 y 50 nodos en cada partici�n
\item Aplicar la heuristica a grafos con menos ejes y entre 50 y 70 nodos en cada partici�n
\end{itemize}

Como en cada experiencia aplicamos las 8 combinaciones, decidimos dividir los gr�ficos, dejando en uno a los que tienen alfa fijo (combinaciones 1,2,3,4) y por otro a los que usan un alfa adaptativo (5,6,7,8) ya que de no hacer esto, se hacia mas dificl visualizar los graficos.

Los resultados de la primer experiencia son los siguientes:
\begin{figure}[H]
\centering
\setcounter{subfigure}{0}
\subfigure[]{
\includegraphics[scale=0.6]{./graficos/grasp/test2.png}}
\setcounter{subfigure}{1}
\subfigure[]{
\includegraphics[scale=0.65]{./graficos/grasp/test22.png}}
\caption{Mejora con respecto a la soluci�n propuesta por la busqueda local}
\end{figure}

\begin{figure}[H]
\centering
\setcounter{subfigure}{0}
\subfigure[]{
\includegraphics[scale=0.6]{./graficos/grasp/tiempos2.png}}
\setcounter{subfigure}{1}
\subfigure[]{
\includegraphics[scale=0.6]{./graficos/grasp/tiempos22.png}}
\caption{Tiempo de ejecuci�n (en segundos)}
\end{figure}

Lo que podemos observar es que si bien no existe uno que se destaque por sobre el resto, en general, 2 y 6 obtienen buenos resultados, lo cual es interesante, si tenemos en cuenta que son metodos que utilizan el criterio de parada adapatativo. Con respecto a los tiempos de ejecuci�n, el criterio adaptativo suele tener tiempos mas bajas, sin embargo en los casos donde mejora poca, por ejemplo, para n=41, el metodo 4 casi no logr� mejoras y como podemos observar su tiempo fue mas alto en ese caso que el de los metodos no adaptativos.

\begin{figure}[H]
\centering
\setcounter{subfigure}{0}
\subfigure[]{
\includegraphics[scale=0.65]{./graficos/grasp/crucesP.png}}
\setcounter{subfigure}{1}
\subfigure[]{
\includegraphics[scale=0.65]{./graficos/grasp/crucesP2.png}}
\caption{Mejora con respecto a la soluci�n propuesta por la busqueda local}
\end{figure}

\begin{figure}[H]
\centering
\setcounter{subfigure}{0}
\subfigure[]{
\includegraphics[scale=0.65]{./graficos/grasp/tiemposP.png}}
\setcounter{subfigure}{1}
\subfigure[]{
\includegraphics[scale=0.65]{./graficos/grasp/tiemposP2.png}}
\caption{Tiempo de ejecuci�n (en segundos)}
\end{figure}

En esta experiencia, se nota claramente la diferencia de tiempo entre los metodos adaptativos y el resto. Con respecto a la mejora en la cantidad de cruces, en este experimento si se observa un m�todo que se desempe�� mejor, el m�todo n�mero 6,  mientras que el m�todo 2 que en la experiencia anterior se habia comportado bastante bien, en esta no lo hizo tan bien.

\subsection{Conclusiones}
A partir de las experiencias, lo que observamos es que si bien hay diferencias entre las distintas combinaciones de parametros, en general no existe un ganador contundente. 

No obstante, en ambas experiencias la combinaci�n \textit{alfa adaptativa, primera posici�n y parada adaptativa} se mostr� como una buena opci�n. Tanto a nivel de mejora en la cantidad de cruces, como a nivel de tiempo de ejecuci�n.

Lo que nos sorprendi�, es el hecho de que utilizar una posici�n aleatoria en lugar de la primer posici�n, no lograra mejoras. 

Por esa raz�n, es que decidimos utilizar esos parametros.

\section{Pseudocodigo}

\begin{algorithm}[H]
\caption{Propone un dibujo mediante la metahuristica GRASP}
\begin{algorithmic}[1]
\STATE solActual $\leftarrow$ construir soluci�n mediante la heursitca constructiva y mejorarla mediante la busqueda local.
\STATE crucesActual $\leftarrow$ cantidad de cruces de la soluci�n propuesta
\STATE iteraciones $\leftarrow$ 0
\STATE maxIteraciones = cantidad de nodos
\STATE alfa $\leftarrow$ 0.95
\WHILE{iteraciones < maxIteraciones}
\STATE nuevoDibujo $\leftarrow$ construir un dibujo con la heuristica constructiva randomizada con alfa, y aplicar busqueda local
\STATE nuevosCruces $\leftarrow$ cantidad de cruces de nuevoDibujo
\IF{ nuevosCruces $<$ crucesActual}
\STATE solActual $\leftarrow$ nuevoDibujo
\STATE crucesActual $\leftarrow$ nuevosCruces
\STATE maxIteraciones $\leftarrow$ maxIteraciones $/$ 2
\ELSE
\STATE iteraciones $\leftarrow$ iteraciones + 1
\STATE alfa $\leftarrow$ minimo(alfa - 0.02,0)
\ENDIF
\ENDWHILE
\RETURN solActual
\end{algorithmic}
\end{algorithm} 

\section{Calculo de complejidad}
Lo primero que hacemos es crear una primer soluci�n mediante la heursitica constructiva y mejorarla con nuestra heuristica de busqueda local. El orden de hacer esto es $O(v_{max}^2*m*log(v_{max})*m^2 + Moviles*v_{max}^2 + m*log(fijos_{max})+fijos_{max} + (V_1+V_2+m)))$. 

Contar los cruces de esta soluci�n tiene un costo $O(m*log(v_max))$, pero este costo es absorvido por la construcci�n de la soluci�n inicial.

Luego comenzamos a iterar. Cada iterac��n tiene el costo de las heuristicas, mas el conteo de cruecs, por lo que vimos recien en total es  $O(v_{max}^2*m*log(v_{max})*m^2 + Moviles*v_{max}^2 + m*log(fijos_{max})+fijos_{max} + (V_1+V_2+m))$. Esto lo hacemos cada vez que iteramos. En el peor de los casos, nunca logramos hacer ninguna mejora y por lo tanto iteramos tantas veces como nodos hay, es decir, $O(v_max)$ iteraciones. Luego el costo total de la heuristica grasp es:

$$O(v_{max}*(v_{max}^2*m*log(v_{max})*m^2 + Moviles*v_{max}^2 + m*log(fijos_{max})+fijos_{max} + (V_1+V_2+m)))$$

Hay que notar que en un mejor caso, siempre mejora por lo que la cantidad de iteraciones no es lineal en $v_{max}$, sino de orden logaritmico.

En funci�n del tama�o de la entrada, sabemos que: 
$$ t = log(P_1)+ \sum_{i=1}^{P_1}{log((p_1)_i)}+ log(P_2)+ \sum_{i=1}^{P_2}{log((p_2)_i)} + log(m_p) + \sum_{i=1}^{m_p}{log((e_i)_0) + log((e_i)_1)} $$
 $$+log(IV_1) + \sum_{i=1}^{IV_1}{log((iv_1)_i)} + log(IV_2) + \sum_{i=1}^{IV_2}{log((iv_2)_i)} + log(m_{iv})+ \sum_{i=1}^{m_{iv}}{log((e'_i)_0) + log((e'_i)_1)} $$ 

Usando esto, mas el calculo hecho para la complejidad de la busqueda local en funci�n de la entrada, podemos ver que el orden es $O(t^6*log(t))$

\section{Analisis experimental}
\subsection{Mal caso}
Para determinar un mal caso para nuestra heuristica Grasp, lo que tenemos que buscar es alg�n mal caso de la heuristica constructiva, que la heuristica de busqueda local no pueda resolver correctamente. Con un caso alcanza, porque si bien la selecci�n de nodos aleatorios, podemos suponer que en el peor caso siempre se repite este ordenamiento malo de los nodos.

Consideremos entonces el ejemplo de caso malo para la constructiva. Recordemos como era:

\begin{figure}[H]
\centering
\setcounter{subfigure}{0}
\includegraphics[scale=0.25]{./figuras/constructivas/malCasoConstructivo.png}
\caption{Mal caso para la heuristica constructiva}
\end{figure}

Ahora apliquemos la busqueda local para ver que resultado obtenemos:

\begin{figure}[H]
\centering
\setcounter{subfigure}{0}
\subfigure[Partimos del dibujo que produce la heuristica constructiva]{
\includegraphics[scale=0.2]{./figuras/constructivas/malCons4.png}}
\subfigure[Para los nodos fijos, no se puede hacer nada. Al nodo 1 lo cambiamos de posici�n pero no porque baja la cantidad de cruces, sino porque en caso de empate, la busqueda local elige la primer posici�n visitada]{
\includegraphics[scale=0.2]{./figuras/grasp/malGrasp1.png}}
\subfigure[Al nodo 2 no se lo mueve porque no cambia el n�mero de cruces. El nodo 3 tampoco cambia el n�mero de cruces, pero se lo mueve por lo dicho antes del criterio de elecci�n de posiciones]{
\includegraphics[scale=0.2]{./figuras/grasp/malGrasp2.png}}
\subfigure[Finalmente se mueve al nodo 4 pero tampoco baja el n�mero de cruces]{
\includegraphics[scale=0.2]{./figuras/grasp/malGrasp3.png}}
\end{figure}

Como el n�mero de cruces no cambio, consideramos que la busqueda local llego a un m�nimo local y no se vuelve a intentar mejorar al dibujo. Sin embargo como vimos en \ref{mal-caso} el dibujo se pod�a lograr con 0 cruces.

Entonces si consideramos la misma familia que hacia fallar a la heuristica constructiva, observamos que la busqueda local no logra mejorar los dibujos que aquella genera, de modo que el Grasp fallar�a de la misma manera que la heuristica constructiva frente a estos casos.

Si bien es cierto que en el peor caso siempre se elige de la misma manera a los nodos a insertar, hay que considerar que en la practica, con un alfa suficientemente bajo como para dar una lista de candidatos adecuadamente grande, es poco probable que se repita siempre la peor elecci�n.
 
\subsection{Comparativa de heuristicas}
\section{Discusi�n}

\newpage

\chapter{Conclusi�n}
\section{Posibles extensiones}
Por razones de tiempo, quedaron diversas experiencias y optimizaciones a los algoritmos 
presentados que no se pudieron realizar. Entre estas contamos:
\begin{itemize}
\item Desarrollar otros algoritmos que nos permitan contar cruces mas r�pidamente
\item Buscar criterios m�s sofisticados de poda para el algoritmo de \textit{backtracking}
\item Intentar disminuir la complejidad espacial y temporal del \textit{backtracking}, principalmente 
      en lo que se refiere al c�lculo de la cota inferior
\item Proponer nuevas heur�sticas constructivas y de b�squeda local que pudieran ser 
      combinadas con las presentadas en este informe para evitar los casos patol�gicos.
\item Realizar un estudio m�s extensivo sobre los diversos par�metros que hacen al funcionamiento 
      del GRASP, principalmente en lo relacionado con el criterio de parada, a fin de optimizar 
      su rendimiento para distintos escenarios (mayor necesidad de velocidad, de precisi�n, o
      alguna relaci�n particular entre ambas).
\item Proponer alguna alternativa basada en otra metaheur�stica, como ser un algoritmo gen�tico 
      para enfrentar este mismo problema
\item Dise�ar una estructura m�s eficiente que nos permita mejorar los �rdenes de complejidad
      de los algoritmos heur�sticos.
\end{itemize}

\section{Conclusiones globales}
      Durante este trabajo pudimos desarrollar un algoritmo exacto para la resoluci�n del 
      problema del dibujo incremental de grafos bipartitos. Sin embargo, si bien consideramos que
      este algoritmo se comporta bien dada su naturaleza factorial, no es capaz de resolver instancias 
      medianamente complejas tiempos aceptables, por lo cual propusimos distintas heur�sticas.

      Pudimos estudiar estas heuristicas, considerando la relaci�n del costo temporal contra la
      optimalidad de los resultados. Constru�mos adem�s un algoritmo basado en la metaheur�stica 
      GRASP, el cual mostr� dar buenos resultados en un tiempo razonable, a�n para instancias muy grandes
      del problema.

      En general tuvimos que afrontar problemas de organizaci�n que se hicieron presentes dada
      la dificultad del problema. Con problemas de estas caracter�sticas, puede resultar impredecible
      el tiempo necesario para implementar un algoritmo heur�stico y obtener resultados
      correctos. Desde este punto de vista, fue muy positivo realizar primero prototipos en un
      lenguaje de alto nivel, lo cual nos dio la libertad de hacer pruebas con mayor
      rapidez. 
      
      Teniendo todo esto en cuenta, consideramos que el trabajo fue interesante porque nos
      permiti� abordar una soluci�n a un problema complejo mediante algoritmos heur�sticos, y
      familiarizarnos con las dificultades relacionadas con este tipo de soluciones.

\begin{thebibliography}{15}
\bibitem{acumTree} Simple and Efficient Bilayer Cross Counting, Wilhelm Barth, Petra Mutzel, Michael Junger, Journal of Graph Algorithms and Applications, vol. 8, no. 2, pp. 179-194 (2004).
\bibitem{usosDelProb} A New Lower Bound for the Bipartite Crossing Number with Applications, Farhad Shahrokhi, Ondrej S�kora, L�szl� Sz�kely, Imrich Vrt`o %si, se llama con una coma al reves el tipo
\bibitem{sugiyama} An Efficient Implementation of Sugiyama's Algorithm for Layered Graph Drawing, Markus Eiglsperger, Martin Siebenhaller, Michael Kaufmann. Journal of Graph Algorithms and Applications
http://jgaa.info/ vol. 9, no. 3, pp. 305-325 (2005) 
\bibitem{lower bound} 2-Layer Straightline Crossing Minimization: Performance of Exact and Heuristic Algorithms, Michael J�nger, Petra Mutzel. Journal of Graph Algorithms and Applications http://www.cs.brown.edu/publications/jgaa/ vol. 1, no. 1, pp. 1-25 (1997)
\bibitem{otroPaper} Heuristics, Experimental Subjects, and Treatment Evaluation in Bigraph Crossing Minimization. Matthias Stallmann, Franc Brglez, and Debabrata Ghosh. North Carolina State University
\bibitem{usoEnDataMinig}cHawk: An Efficient Biclustering Algorithm based on Bipartite Graph Crossing Minimization. Waseem Ahmad, Ashfaq Khokhar. %FIXME: si no se habla de data mining en la intro, fletar esta cosa.
\bibitem{MaxCut} Fixed Linear Crossing Minimization by Reduction to the Maximum Cut Problem. Christoph Buchheim and Lanbo Zheng. %FIXME: de aca se pueden sacar ideas sobre para q corno sirve el problema. Por ej en las ref 11 y 3 en el paper, pero no las encontre en la web.
\end{thebibliography}
\label{LastPage}
\end{document}
