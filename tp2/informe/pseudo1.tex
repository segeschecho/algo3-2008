\begin{algorithm}
\caption{Devuelve la lista de aquellos jugadores, tal que se puede arreglar el torneo}
\begin{algorithmic}[1]
\STATE fuertes $\leftarrow$ armarFuertes(grafo)
\COMMENT {Averiguamos en que componente quedo cada nodo}
\FOR {i $\in$ {$1,...,$ Cantidad de componentes fuertemente conexas}}
	\FOR {cada nodo $\in$ $fuertes_i$}
			\STATE dondeQuedo[nodo] $\leftarrow$ i
	\ENDFOR
\ENDFOR
\STATE relacion$\leftarrow[ ]$
\FOR{cada nodo del grafo}
	\FOR{cada nodo2 que llega al nodo}
			\IF{si el vertice no une elementos de la misma componente}
				\STATE relacion $+$ $[(dondeQuedo[nodo],dondeQuedo[nodo2])]$
			\ENDIF
	\ENDFOR
\ENDFOR

\STATE g1 $\leftarrow$ Grafo(cantidad de Componentes, relacion)
\COMMENT{Una vez que tengo el grafo reducido, busco cuantos hay con $d_{in} = 0$}
\STATE encontreUno = false
\STATE quien $\leftarrow$ $\bot$
\FOR{cada nodo de g1}
	\IF{ $d_{in}(nodo) == 0$ $\wedge$ no encontreUno}
		\STATE quien $\leftarrow$ nodo
		\STATE encontreUno $\leftarrow$ True
	\ELSIF{ $d_{in}(nodo) == 0$ $\wedge$ encontreUno}
		\STATE devolver $[]$
	\ENDIF
\ENDFOR
\STATE $ordenar(fuertes[quien])$ \COMMENT{lo hago con bucket sort, ya que se que estan entre 0 y cantidad de nodos del grafo original}
\STATE devolver $fuertes[quien]$	

\end{algorithmic}
\end{algorithm}
