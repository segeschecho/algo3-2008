\part{Ejercicio 1}
\section{Enunciado}
Un torneo de Tenis de eliminaci'on simple consiste en varios partidos donde el perdedor de
cada partido es eliminado del torneo y no vuelve a jugar un partido en ese torneo. El fixture
del torneo se arma al comienzo del mismo tomando dos jugadores a'un no eliminados para
cada partido, hasta que quede s'olo un jugador no eliminado, que resulta ser el ganador.

Con este esquema de fixture no s'olo la destreza o el entrenamiento entran en juego para
decidir el ganador sino que la suerte tiene un papel importante.

Despu'es de observar el entrenamiento de los participantes hay ciertos partidos de los
cuales se puede saber con certeza su resultado, es decir, para ciertos jugadores a,b, se
puede asegurar que a le gana a b.

Diremos que el torneo puede ser arreglado para que gane x si existe un fixture de elimi
naci'on simple donde se puede asegurar que gane x.
Encontrar todos los participantes x para los cuales el torneo puede ser arreglado para gane
x.

Modelar este problema utilizando grafos. Justificar el modelo.

El mejor algoritmo que conocemos es de O(n + m).

\section{Desarrollo}
\subsection{Sobre el modelo}
Para resolver el problema, planteamos un d�grafo donde cada nodo es un jugador y a$\leadsto$b si  a le gana a b. 

Lo primero que observamos es que la relaci�n de ganar, por como esta planteada, es transitiva. Decimos que es transitiva en el sentido de que si a$\leadsto$b y b$\leadsto$c, se puede hacer que a le gane a c, haciendo que b le gane a c y luego a le gane a b. Es decir, si bien a$\leadsto$c, se puede organizar los partidos para que a pueda ganar.

Entonces seg�n este modelo, un jugador, jugador A, puede ganar si para cada uno de los otros jugadores, jugador B, existe un 
camino dirigido que comunica a A con B. Luego si usamos esto, podemos obtener una primera forma de resolver el problema: Para 
cada nodo, tratamos de recorrer todo el grafo. Si lo logramos, se puede armar un torneo para que gane. Como esto se hace para 
cada nodo, el orden queda $O(n*m)$.

Buscamos entonces alguna forma de mejorar el orden. Una primera alternativa era guardar 
alguna informaci�n en cada recorrida, para no repetir c�lculos, pero los ciclos nos imped�an lograr alguna soluci�n. Tuvimos  entonces que buscar alguna otra forma.

\subsection{Resolucion en grafos aciclicos}
Si el grafo que se obtiene no presenta ciclos, se puede ver que existe un ganador si y solo si existe un �nico nodo tal que su 
grado de entrada es 0. Por ejemplo, el grafo de la figura ~\ref{fig:conGanador}:

\begin{figure}[H]
\centering
\includegraphics[scale=0.5]{./figuras/conGanador.png}
\caption{Ejemplo con un solo nodo con grado de entrada nulo}
 \label{fig:conGanador}
\end{figure}
En este caso, solo puede haber un ganador, el jugador 1.

Si existen varios, no se podr�n eliminar nunca, ya que no hay quien les gane con seguridad. Por ejemplo, lo grafos de la figura ~\ref{fig:sinGanador}:

\begin{figure}[H]
    \begin{minipage}{.5\linewidth}
    \centering
     \includegraphics[scale=0.5]{./figuras/sinGanador1.png}
    \end{minipage}
    \begin{minipage}{.5\linewidth}
    \centering
      \includegraphics[scale=0.5]{./figuras/sinGanador2.png}
    \end{minipage}
 \caption{Ejemplos con mas de un nodo con grado de entrada nulo}
 \label{fig:sinGanador}
\end{figure}
\afterpage{\clearpage}
En el primer caso, el 5 y el 1 no se pueden eliminar. An�logamente, en el segundo caso, 1 y 6 no se pueden eliminar en ning�n momento.
Por otro lado, no puede ocurrir que no halla ningun nodo con grado de entrada 0, porque si eso ocurre existe un ciclo. (Ver 
demostraci�n 1).

Entonces si no hay ciclos, podemos resolver f�cilmente el problema: Si representamos el grafo con listas de adyacencia (tanto 
de entrada, como de salida), podemos mirar cada nodo, viendo si hay solo uno con grado de entrada 0. Si es as�, ese gana. Si 
encontramos varios, no existe ganador. Esto tiene como costo O(n), ya que recorremos todos los nodos, y preguntamos cuanto mide 
su lista de adyacencia de salida (costo constante).

Ahora bien, no se puede afirmar que el grafo que recibimos no presente ciclos. Es decir hay instancias validas que poseen 
ciclos, por ejemplo la figura ~\ref{fig:conCiclo}:

\begin{figure}[H]
\centering
\includegraphics[scale=0.5]{./figuras/conCiclo.png}
\caption{Ejemplo sin ningun nodo con grado de entrada 0}
 \label{fig:conCiclo}
\end{figure} 
En este caso, el torneo puede arreglarse tanto para 1, como para 2,4 o 5. Debemos entonces buscar alguna manera de salvar esta dificultad.

\subsection{El papel de las componentes fuertemente conexas}
Si tenemos ciclos, no vale la propiedad antes enunciada sobre los grados de entrada.

Analizamos entonces que ocurre si el grafo presenta un ciclo. Como la relaci�n a$\leadsto$b es transitiva (en el sentido que 
comentamos antes), todos los elementos que pertenecen a un ciclo, se ganan entre si. Por otro lado, si tomamos un elemento que 
no este en el ciclo (a) y que le gane a alguien del mismo (b), podemos ver que les puede ganar a todos: primero hacemos que b 
le gane a todos los del ciclo, y luego hacemos que a le gane a b. An�logamente se puede ver que si alguien del ciclo, le gana a 
alguien que no esta en el; cualquier otro del ciclo le puede ganar.

Esto nos hace pensar que podemos considerar al cada ciclo como una unidad, como un jugador �nico, que les gana a todos 
aquellos que son derrotados por alg�n individuo del ciclo y que pierde contra todos aquellos que le ganan a alguien del ciclo.

\begin{figure}[H]
    \begin{minipage}{.5\linewidth}
    \centering
     \includegraphics[scale=0.5]{./figuras/componentesConexas.png}
    \end{minipage}
    \begin{minipage}{.5\linewidth}
    \centering
      \includegraphics[scale=0.7]{./figuras/reducido.png}
    \end{minipage}
 \caption{Ejemplo de reducci'on de un grafo}
 \label{fig:reduccion}
\end{figure}
%\afterpage{\clearpage}

Estos ciclos que buscamos no son mas que las componentes fuertemente conexas del grafo. Si reducimos al grafo de modo de que 
colapsamos a los nodos que pertenecen a una componente fuertemente conexa a un �nico nodo (como muestra la figura: ~\ref{fig:reduccion}), actualizando la informaci�n de los 
partidos arreglados, lo que obtenemos es un nuevo grafo que cumple ser libre de ciclos (ver demostracion 2).

Entonces una vez que tenemos un grafo libre de ciclos, podemos aplicar la propiedad que enunciamos antes, y resolver el problema en $O(n)$.


\subsection{Obtenci�n de las componentes fuertemente conexas}
Para poder eliminar los ciclos, buscamos las componentes fuertemente conexas. Para hacerlo utilizamos el algoritmo de Kosaraju. 
El mismo logra encontrarlas en O(n+m). El algoritmo es b�sicamente DFS.

Funciona de la siguiente manera:
\begin{itemize}
\item Primero realiza un DFS numerando los nodos seg�n el orden de finalizaci�n de las llamadas recursivas. (se repite hasta numerar todos los nodos).

\item Luego se arma el grafo $g'$ que contiene los mismos nodos que g pero a$\leadsto$b en $g'$ si y solo si b$\leadsto$a en g.

\item Una vez armado $g'$, se realiza un DFS en 'el, partiendo del nodo con mayor numeraci'on. Al terminar se obtiene una componente fuertemente conexa. 

\item El proceso se repite para todos los nodos no visitados, siempre en orden decreciente de numeraci'on.

\end{itemize}
Como lo que hace el algoritmo es DFS dos veces, tiene orden O(n+m)

Por ejemplo, apliquemos el algoritmo al grafo de la figura ~\ref{fig:kosaraju1}:

\begin{figure}[H]
\centering
\subfigure[Comenzamos con el grafo original]{
\includegraphics[scale=0.6]{./figuras/kosaraju1.png} }\hspace{1in} 
\subfigure[Realizamos DFS partindo desde el 1 y numeramos los nodos segun el orden de la llamada recursiva]{
\includegraphics[scale=0.6]{./figuras/kosaraju2.png}}
\subfigure[Como el 8 quedo sin marcar, realizamos una segunda DFS partiendo desde el]{
\includegraphics[scale=0.6]{./figuras/kosaraju3.png}}\hspace{1in} 
\subfigure[Una vez que marcamos todos los nodos, invertimos el grafo]{
\includegraphics[scale=0.6]{./figuras/kosaraju4.png}} 
\subfigure[Como el 8 es el elemento de mayor numeraci'on comenzamos por �l. Hacemos DFS, y todos los nodos que tocamos, son una componente fuertemente conexa]{
\includegraphics[scale=0.6]{./figuras/kosaraju5.png} }\hspace{1in} 
\subfigure[Ahora seguimos con el 1, el de mayor numeraci'on sin visitar. Luego de hacer DFS tenemos otra componente fuertemente conexa de G. Como no quedan mas nodos por visitar, terminamos]{
\includegraphics[scale=0.6]{./figuras/kosaraju6.png}}
\caption{Algoritmo de Kosaraju}
\label{fig:kosaraju1}
\end{figure}

\subsection{Armado del grafo reducido y resoluci�n del problema}
Una vez que ya tenemos las componentes fuertemente conexas, armar el grafo reducido es simple. 
\begin{itemize}
\item Primero guardamos en que componente quedo cada jugador

\item Luego tomamos las relaciones entre los jugadores, y las traducimos al nuevo grafo: las 
releciones dentro de la misma componente se descartan, y dos componentes estan relacionadas si existe un jugador en cada una, 
tal que esten relacionados.Es de notar que es necesario filtrar las relaciones intracomponente para no obtener un 
pseudografo que no nos permite usar la propiedad de los grafos sin ciclos, pues un nodo queda relacionado con si mismo, por lo 
que tiene grado de entrada mayor  a 0. Tambien es de notar que no podemos obtener un multigrafo, ya que el grafo es libre de 
ciclos: si vale que a$\leadsto$b y b$\leadsto$a, a y b deberian ser una unica componente fuertemente conexa.

\item Con esta informaci�n podemos armar el nuevo grafo.

\item Contamos cuantos elementos tienen grado de llegada 0, si solo hay uno ese gano. Entonces ganan todos los elementos de la componente. Como podrian no estar en orden, las ordenamos mediante bucket sort en $O(n)$

\end{itemize}

\subsection{Demostraciones auxiliares}
\newtheorem{teorema 1}{Teorema}
\begin{teorema 1}
\normalsize
Sea g=(V,X) tal que $\forall$ v $\in$ V, $d_{in}(v)$ $>$ 0 $\Leftarrow$ existe un ciclo en g
\end{teorema 1}

\renewcommand*{\proofname}{Demostraci�n}

\begin{proof}
\normalsize
Sea g=(V,X) tal que  $\forall$ v $\in$ V, d(v) $>$ 0, consideremos el camino maximo de g, $v_1,v_2,...v_i,...,v_n$
\begin{figure}[H]
\centering
\includegraphics[scale=0.5]{./figuras/demostracion.png}
\end{figure} 
Pero como f $d_{in}(v)$ $>$ 0, existe $v_k$ tal que $v_k$ $\leadsto$ $v_1$. Si $v_k$ pertenece al camino, tenemos un ciclo, que era lo que quer'iamos demostrar.
Supongamos que no pertenece al camino. Entonces tengo un camino nuevo, que va de $v_k$ a $v_n$, que tiene mayor longitud que
el camino de $v_1$ a $v_n$, absurdo, puesto que $v_1,v_2,...v_i,...,v_n$ era el camino m�ximo.
\end{proof}
\vspace{0.2in}
\begin{teorema 1}
\normalsize
Sea g un grafo reducido, es decir que cada nodo es una componente fuertemente conexa, entonces no existen circuitos en g
\end{teorema 1}
\begin{proof}
\normalsize
Supongamos que existe un circuito en g. Sean $a_1$,$a_2$,..., $a_i$,..., $a_n$ nodos, tal que el circuito pasa por ellos. Como tengo un circuito vale que para cada par de nodos $a_i$, $a_j$, existe un camino. Entonces $a_1$, $a_2$, ..., $a_n$ forman una componente fuertemente conexa. Absurdo, que provino de suponer que existia un circuito en g.
\end{proof}

\newpage
\section{Pseudocodigo}
\begin{algorithm}
\caption{Devuelve la lista de aquellos jugadores, tal que se puede arreglar el torneo}
\label{alg:algoritmo1}
\begin{algorithmic}[1]
\PARAMS {Grafo con jugadores como nodos, y partidos arreglados como aristas direccionadas}
\IF{hay menos partidos arreglados que jugadores \textcolor{orange}{$-$} 1}
	\RETURN \textcolor{orange}{[ ]}
\ENDIF
\STATE fuertes $\textcolor{orange}{\leftarrow}$ armarFuertes\textcolor{SkyBlue}{(}grafo\textcolor{SkyBlue}{)}
\COMMENT {Averiguamos en que componente quedo cada nodo}
\FOR {i \textcolor{orange}{$\in$} {$1,...,$ Cantidad de componentes fuertemente conexas}}
	\FOR {cada nodo \textcolor{orange}{$\in$} $fuertes_i$}
			\STATE dondeQuedo\textcolor{orange}{[}nodo\textcolor{orange}{]} $\textcolor{orange}{\leftarrow}$ i
	\ENDFOR
\ENDFOR
\STATE relacion$\textcolor{orange}{\leftarrow}\textcolor{orange}{[} \textcolor{orange}{ ]}$
\FOR{cada nodo del grafo}
	\FOR{cada nodo2 al que llega nodo}
			\IF{si el vertice no une elementos de la misma componente}
				\STATE relacion $\textcolor{orange}{+}$ $\textcolor{orange}{[ }\textcolor{SkyBlue}{(}dondeQuedo\textcolor{orange}{[}nodo\textcolor{orange}{]},dondeQuedo\textcolor{orange}{[}nodo2\textcolor{orange}{]}\textcolor{SkyBlue}{)}\textcolor{orange}{ ]}$
			\ENDIF
	\ENDFOR
\ENDFOR
\STATE g1 $\textcolor{orange}{\leftarrow}$ Grafo\textcolor{SkyBlue}{(}cantidad de Componentes, relacion\textcolor{SkyBlue}{)}
\COMMENT{Una vez que tengo el grafo reducido, busco cuantos hay con $d_{in} = 0$}
\STATE encontreUno $\textcolor{orange}{\leftarrow}$ false
\STATE quien $\textcolor{orange}{\leftarrow}$ \textcolor{orange}{$\bot$}
\FOR{cada nodo de g1}
	\IF{ $d_{in}\textcolor{SkyBlue}{(}nodo\textcolor{SkyBlue}{)} \textcolor{orange}{==} 0$ \textcolor{orange}{$\wedge$} no encontreUno}
		\STATE quien $\textcolor{orange}{\leftarrow}$ nodo
		\STATE encontreUno $\textcolor{orange}{\leftarrow}$ True
	\ELSIF{ $d_{in}\textcolor{SkyBlue}{(}nodo\textcolor{SkyBlue}{)} \textcolor{orange}{==} 0$ \textcolor{orange}{$\wedge$} encontreUno}
		\RETURN  $\textcolor{orange}{[ }\textcolor{orange}{ ]}$
	\ENDIF
\ENDFOR
\STATE $ordenar\textcolor{SkyBlue}{(}fuertes\textcolor{orange}{[}quien\textcolor{orange}{]}\textcolor{SkyBlue}{)}$ \COMMENT{lo hago con bucket sort, ya que se que estan entre 1 y cantidad de nodos del grafo original}
\RETURN $fuertes\textcolor{orange}{[}quien\textcolor{orange}{]}$	
\end{algorithmic}
\end{algorithm}

\begin{algorithm}
\caption{Devuelve la lista de las componentes fuertemente conexas mediante algoritmo de Kosaraju}
\label{alg:algoritmo2}
\begin{algorithmic}[1]
\PARAMS{Grafo con jugadores como nodos, y partidos arreglados como aristas direccionadas}
\STATE valor $\textcolor{orange}{\leftarrow}$ {0...0}
\STATE dar valor a todos los nodos segun las llamadas recursivas al hacer dsf
\STATE en valor\textcolor{orange}{[}i\textcolor{orange}{]} colocar al nodo que tiene valor i
\STATE $visitado \textcolor{orange}{\leftarrow} {0...0}$
\STATE g $\textcolor{orange}{\leftarrow}$ invertirGrafo\textcolor{SkyBlue}{(}grafo\textcolor{SkyBlue}{)}
\STATE $CompFuertes \textcolor{orange}{\leftarrow} \textcolor{orange}{[}\textcolor{orange}{]}$
\FOR{cada nodo con mayor valor y sin visitar}
		\STATE fuerte $\textcolor{orange}{\leftarrow}$ \textcolor{orange}{[}\textcolor{orange}{]}
		\STATE Realizar dsf desde el nodo, los elementos visitados son una componente fuertemente conexa
		\STATE $CompFuertes$ $\textcolor{orange}{\leftarrow}$ $CompFuertes$ \textcolor{orange}{+} \textcolor{orange}{[}fuerte\textcolor{orange}{]}
\ENDFOR
\RETURN fuertes
\end{algorithmic}
\end{algorithm}

\begin{algorithm}
\caption{Genera el grafo inverso de un grafo dado}
\label{alg:algoritmo3}
\begin{algorithmic}
\PARAMS{Grafo que se desea invertir}
\STATE relaciones $\textcolor{orange}{\leftarrow}$ [ ]
\FOR{ cada nodo a del grafo}
	\FOR{ cada otro nodo b del grafo, tal que a$\leadsto$b}
		\STATE agregar a relaciones (b,a)
	\ENDFOR
\ENDFOR
\RETURN Grafo(cantidad de nodos del grafo, relaciones)	
\end{algorithmic}
\end{algorithm}

No se incluyen los algoritmos que usan dsf, ya que hacen exactamente dsf y, o bien numeran los nodos o bien los agregan a una lista. Creemos que no es necesario el pseudocodigo de dichos algoritmos.
			
%\begin{algorithm}
%\caption{numera los nodos mediante dsf}
%\begin{algorithmic}\textcolor{orange}{[}1\textcolor{orange}{]}
%\STATE $visitado \textcolor{orange}{\leftarrow} {0...0}$
%\FOR{cada nodo del grafo}
%	\IF{no lo visite}
%		\STATE dsf(grafo, nodo,valor) \COMMENT{dsf que numera}		
%	\ENDIF
%	\ENDFOR
%	
%\end{algorithmic}
%\end{algorithm}
%
%\begin{algorithm}
%\caption{Realiz un dsf numerando a cada nodo segun el orden de su llamada, guarda que valor tiene cada nodo}
%\begin{algorithmic}\textcolor{orange}{[}1\textcolor{orange}{]}
%\STATE visitado\textcolor{orange}{[}nodo\textcolor{orange}{]} $\textcolor{orange}{\leftarrow}$ 1
%\FOR{Cada vertice relacionado con el nodo}
%	\IF{no lo visite}
%		\STATE bfp(grafo,vertice}
%	\ENDIF
%\ENDFOR
%\STATE Valor\textcolor{orange}{[}valorActual\textcolor{orange}{]} $\textcolor{orange}{\leftarrow}$ nodo
%\STATE valorActual $\textcolor{orange}{+}\textcolor{orange}{+}$
%

